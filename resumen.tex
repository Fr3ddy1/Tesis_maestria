\chapter*{Resumen.}

\hspace{0.4cm} La curva de rendimientos es una herramienta muy utilizada por las autoridades de los bancos centrales y por los inversionistas al momento de realizar alguna operaci\'on \'o inversi\'on con alg\'un instrumento financiero, ya que la misma refleja el precio intertemporal del dinero. Para su c\'alculo o estimaci\'on existen diversas metodolog\'ias, entre las que se destacan las metodolog\'ias param\'etricas y las no param\'etricas cada una con sus ventajas y desventajas.

\hspace{0.4cm} Uno de los principales usos de esta herramienta es el de estimar los precios te\'oricos de alg\'un instrumento financiero en un instante determinado, particularmente en este trabajo se consideraran los instrumentos de la Deuda P\'ublica Nacional (DPN). Esto con el fin de poder valorar un portafolio de inversiones en un momento dado y determinar as\'i si el mismo genera una ganacia \'o una perdida.

\hspace{0.4cm} En el siguiente trabajo se propone el uso de la metodolog\'ia de los Splines c\'ubicos de suavizado para estimar la curva de rendimientos en un momento determinado y as\'i poder calcular los precios te\'oricos de los instrumentos que se deseen considerar. Esta metodolog\'ia presenta un enfoque no param\'etrico, la misma se caracteriza por trabajar directamente con los datos y por contar con un factor de suavizamiento muy importante al considerar la suavidad de la curva ajustada. 



