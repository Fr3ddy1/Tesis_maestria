\documentclass[12pt, oneside, letterpaper]{amsbook}

%a continuaci?n el paquete que permite incluir gr?ficos
\usepackage{graphics}
\usepackage{amsmath}
\usepackage{chngcntr}
\counterwithout{table}{chapter}

%el siguiente paquete permite acentuar de la manera usual
%se recomienda usarlo porque facilita la correcci?n de la tesis
%con un diccionario de espa?ol
\usepackage[ansinew]{inputenc}
%\usepackage[utf8]{inputenc}

%a continuaci?n el paquete y el comando para que las figuras no se
%ubiquen siempre en la parte de arriba de la p?gina
\usepackage{float}
\floatplacement{figure}{ht}

%el uso del siguiente paquete (babel) va a permitir que las palabras
%se separen correctamente
%no olvide seleccionar "spanish" al principio del documento
\usepackage[spanish,english]{babel}


%el siguiente comando permite la elaboraci?n del ?ndice anal?tico
\makeindex

%el siguiente grupo de comandos fija los m?rgenes del documento

% Set left margin - The default is 1 inch, so the following
\setlength{\oddsidemargin}{0in}
%\setlength{\evensidemargin}{-0.1in}

% Set width of the text - What is left will be the right margin.
\setlength{\textwidth}{6.5in}

% Set top margin - The default is 1 inch, so the following
% command sets a 0.75-inch top margin.
\setlength{\topmargin}{-.4in}

% Set height of the text - What is left will be the bottom margin.
% In this case, bottom margin is 11in - 0.75in - 9.5in = 0.75in
\setlength{\textheight}{9in}

% Set height of the header
%\setlength{\headheight}{0.3in}

% Set vertical distance between the header and the text
\setlength{\headsep}{0.4in}

% Set vertical distance between the text and the
% bottom of footer
\setlength{\footskip}{0.4in}

%el siguiente comando define el espacio entre lineas
\renewcommand{\baselinestretch}{1.5}

%los siguientes comando cambian los nombres "Chapter", "Proof", etc
%a "Cap?tulo", "Demostraci?n", etc.
%NO hacen falta si usa el paquete "babel"

\renewcommand{\chaptername}{Cap\'{\i}tulo}
\renewcommand{\contentsname}{CONTENIDO}
\renewcommand{\proofname}{Demostraci\'on}
\renewcommand{\bibname}{Bibliograf\'{\i}a}
\renewcommand{\indexname}{\'Indice}
\renewcommand{\figurename}{Figura}
%\renewcommand{\tablename}{Tabla}

%el siguiente comando define "seno" como funci?n (esta funci?n
%no la trae LaTeX porque es ingl?s es "sine")
%igual para arcoseno
\newcommand{\sen}{\operatorname{sen}}
\newcommand{\arcsen}{\operatorname{arcsen}}

\theoremstyle{plain}
\newtheorem{theorem}{Teorema}[chapter]
\newtheorem{proposition}[theorem]{Proposici\'on}
\newtheorem{lemma}[theorem]{Lema}
\newtheorem{corollary}[theorem]{Corolario}


\theoremstyle{definition}
\newtheorem{definition}[theorem]{Definici\'on}
\newtheorem{example}[theorem]{Ejemplo}



\theoremstyle{remark}
\newtheorem{remark}[theorem]{Observaci\'on}

\numberwithin{equation}{chapter} \numberwithin{figure}{chapter}


%Optativo
%el siguiente paquete hace que el documento tenga hiperv?nculos
%es muy ?til a la hora de tipear y hace la versi?n electr?nica
%mucho m?s atractiva
%se debe colocar justo antes de "\begin{document}"

%\usepackage[letterpaper,backref,plainpages=false,pagebackref]{hyperref}


\begin{document}

\selectlanguage{spanish}

\frontmatter


\thispagestyle{empty}


\begin{center}

\vspace*{-1.5cm}

\scalebox{.12}{\includegraphics{ucv.eps}}

\vspace{.5cm}

\begin{small}
UNIVERSIDAD CENTRAL DE VENEZUELA

\vspace{-0.1cm}

FACULTAD DE CIENCIAS

\vspace{-0.1cm}

ESCUELA DE MATEM\'ATICA

\vspace{-0.1cm}

MAESTR\'IA EN MODELOS ALEATORIOS

\vspace{-0.1cm}

\end{small}

\vspace{2cm}

\begin{Huge}

{\bf Estimaci\'on de la curva de rendimientos mediante}

{\bf el uso de splines c\'ubicos suavizados}


\end{Huge}

\end{center}

\vspace{2cm}

\hspace{6cm}
\begin{minipage}[t]{9cm}
%\begin{large}
Trabajo de Grado de Maestr\'ia presentado ante la ilustre Universidad Central de Venezuela por el
\textbf{Lic. Freddy F. Tapia C.} para optar al t\'{\i}tulo de \textbf{Magister Scientiarium Menci\'on Modelos Aleatorios}

\

\textbf{Tutor: Dr. Ricardo Rios.}
%\end{large}
\end{minipage}



\vspace{1.5cm}

\begin{center}
Caracas, Venezuela \\
Noviembre y 2018
\end{center}

\newpage


%%%%NO USAR EN TESIS DE POSTGRADO%%%%%%%%


Nosotros, los abajo firmantes, designados por la Universidad Central de Venezuela como
integrantes del Jurado Examinador del Trabajo Especial de Grado titulado ``\textbf{Nombre del
Trabajo Especial de Grado}'', presentado por el \textbf{Br. Nombre del Estudiante}, titular de
la C\'edula de Identidad \textbf{XXXXXXXXX}, certificamos que este trabajo cumple con los
requisitos exigidos por nuestra Magna Casa de Estudios para optar al t\'{\i}tulo de
\textbf{Licenciado en Matem\'atica}.

%%%%NO USAR EN TESIS DE POSTGRADO%%%%%%%%


\vspace{2cm}

\begin{center}
\underline{\hspace{8cm}}

\vspace{1cm}

\textbf{ Nombre del Tutor \\Tutor}

\vspace{3cm}

\underline{\hspace{8cm}}

\vspace{1cm}

\textbf{ Nombre del Jurado \\Jurado}

\vspace{3cm}

\underline{\hspace{8cm}}

\vspace{1cm}

\textbf{ Nombre del Jurado \\Jurado}

\end{center}

\newpage


\vspace*{2.5cm}

\begin{center}
{\large Dedicatoria}
\end{center}
\vspace{1cm}

\hspace*{0.4 cm} A mis padres, Luis F. Tapia y Nancy M. Calvache por su amor y apoyo incondicional durante todo el tiempo que trancurri\'o esta maestr\'ia y por todos los a\~nos antes de ella, soy afortunado en tenerlos.

\hspace*{0.4 cm} A mis hermanas, Mayra y Vanessa Tapia por su apoyo y afecto durante todo este tiempo, cada una mostrandome su afecto a su manera, gracias por todos esos momentos felices.

\hspace*{0.4 cm} A mis primos, tios y abuela por sus consejos y por estar ahi cuando los necesito y por poder contar siempre con ustedes.

\hspace*{0.4 cm} En especial a mi abuelo Luis H. Calvache, que me hace mucha falta pero s\'e que desde el cielo siempre me apoya.

\newpage


\vspace*{2.5cm}

\begin{center}
{\large Agradecimiento}
\end{center}
\vspace{1cm}

Colocar aqu\'{\i} el agradecimiento (optativo).

\newpage



\tableofcontents

\renewcommand{\listfigurename}{\'{I}ndice de figuras.} 
\listoffigures 

\renewcommand{\listtablename}{\'{I}ndice de tablas.} 
\listoftables 

\mainmatter

\chapter*{Resumen.}

%\hspace{0.4cm} La curva de rendimientos es una herramienta muy utilizada por las autoridades de los bancos centrales y por los inversionistas al momento de realizar alguna operaci\'on \'o inversi\'on con alg\'un instrumento financiero, ya que la misma refleja el precio intertemporal del dinero. Para su c\'alculo \'o estimaci\'on existen diversas metodolog\'ias, entre las que se destacan las metodolog\'ias param\'etricas y las no param\'etricas cada una con sus ventajas y desventajas.

%\hspace{0.4cm} Uno de los principales usos de esta herramienta es el de estimar los precios te\'oricos de alg\'un instrumento financiero en un instante determinado, particularmente en este trabajo se consideraran los instrumentos de la Deuda P\'ublica Nacional (DPN). Esto con el fin de poder valorar un portafolio de inversiones en un momento dado y determinar as\'i si el mismo genera una ganacia \'o una perdida.

\hspace{0.4cm} En el siguiente trabajo se propone el uso de la metodolog\'ia de los Splines c\'ubicos de suavizado para estimar la curva de rendimientos en un momento determinado y as\'i poder calcular los precios te\'oricos de los instrumentos que se deseen considerar. Esta metodolog\'ia presenta un enfoque no param\'etrico, la misma se caracteriza por trabajar directamente con los datos y por contar con un factor de suavizamiento muy importante al considerar la suavidad de la curva ajustada. 

Palabras clave: 


\chapter*{Introducci\'on}

El objetivo de esta plantilla es facilitar el  tipeo de las tesis a los estudiantes de la
Licenciatura en Matem\'atica de la UCV y permitir una presentaci\'on uniforme de estos
trabajos.


\chapter{Antecedentes y motivaci\'on.}

\section{Introducci\'on.}

\hspace*{0.4 cm} La curva de redimientos es una herramienta ampliamente utilizada por las autoridades de los bancos centrales en sus decisiones de pol\'itica monetaria, as\'i como tambi\'en por los agentes privados en la planificaci\'on de sus inversiones en instrumentos financieros [1]. La misma tiene una importancia capital
para el mundo acad\'emico y pr\'actico desde el punto de vista econ\'omico y financiero, al reflejar el precio intertemporal del dinero.


\hspace*{0.4 cm} La curva de rendimientos es una representaci\'on grafica que muestra la relaci\'on que existe entre los rendimientos de una clase particular de t\'itulos valores y el tiempo que falta para su vencimiento, lo cual es conocido como la estructura temporal de la tasa de inter\'es (ETTI) para instrumentos con riesgo similar pero con diferentes plazos de maduraci\'on. La ETTI es un indicador de la evoluci\'on futura de los tipos de inter\'es y de inflaci\'on, adem\'as, la mayor\'ia de los activos financieros se valoran mediante este indicador, por lo cual tambi\'en se considera b\'asico en el dise\~no de estrategias de gesti\'on de riesgos y en la toma de decisiones de inversi\'on y financiaci\'on (Fern\'andez J.L., Robles M.D., 2005, p. 243). Existen cuatro formas que puede adoptar una curva de rendimientos:

\begin{itemize}
  \item Curva ascendente: generalmente, la curva de rendimientos tiene esta forma, lo que indica que los inversionistas requieren mayores rendimientos para vencimientos de m\'as largo plazo, es decir, que los rendimientos var\'ian directamente con los plazos. 
  \item Curva descendente: indica que los rendimientos disminuyen a medida que aumentan los plazos.
  \item Curva horizontal: indica que independientemente del plazo de vencimiento, los rendimientos son los mismos; para per\'iodos muy largos, todas las curvas de rendimientos tienden a aplanarse.
  \item Curva creciente y decreciente: es el reflejo de una situaci\'on en la que los rendimientos de corto y largo plazo son los mismos y los rendimientos de mediano plazo son los que var\'ian.
\end{itemize}

\hspace*{0.4 cm} Es de esperar que una pendiente negativa de la curva de rendimientos o curva invertida (tasas de largo plazo menores a las de corto plazo) indique expectativas de una recesi\'on futura y, por lo tanto, menores tasas de inter\'es futuras; esto se puede explicar ya que los rendimientos esperados contienen informaci\'on sobre los planes de consumo de los agentes. 

\hspace*{0.4 cm} Entre las teor\'ias que explican la pendiente de la curva de rendimientos, se encuentran:

\begin{itemize}
  \item La teor\'ia de la preferencia por la liquidez: consiste en que los inversionistas prefieren manejar t\'i tulos a corto plazo, pues \'estos tienen una sensibilidad menor a los cambios en las tasas de inter\'es y ofrecen una mayor flexibilidad en las inversiones si se compara con los t\'itulos de largo plazo. Adem\'as, los prestatarios prefieren deuda a largo plazo, pues la de corto plazo los expone al riesgo de hacer una refinanciaci\'on de la deuda en condiciones adversas. Ambas situaciones, generan entonces, tasas de corto plazo relativamente bajas. En su conjunto, estos dos grupos de preferencias implican que en condiciones normales existe una Prima de Riesgo por Vencimiento (PRV) que aumenta en funci\'on de los a\~nos de vencimiento, haciendo que la curva de rendimientos posea una pendiente ascendente (Douglas, 1988 pp 367-370).
  \item La teor\'ia de la segmentaci\'on del mercado: considera el mercado de renta fija como una serie de distintos mercados, los inversionistas y los emisores est\'an restringidos por el sector espec\'ifico de maduraci\'on. De acuerdo con esta teor\'ia, la curva de rendimientos refleja una serie de condiciones de oferta y demanda que crean una secuencia de precios de equilibrio de mercado (tasas de inter\'es) de los fondos (Douglas, 1988, pp. 368-369).
  \item La teor\'ia del H\'abitat Preferido: plantea que los inversionistas intentar\'an liquidar sus inversiones en el menor plazo posible mientras que los prestamistas querr\'an tomar un plazo más largo; por lo tanto, dado que no se encuentran oferta y demanda de fondos para un mismo plazo, algunos inversionistas o prestatarios se ver\'an motivados a cambiar el plazo de la inversi\'on o el financiamiento pero, para lograrlo, deben ser compensados con un premio por el riesgo cuyo tamano reflejar\'a la extesi\'on de la aversi\'on al riesgo.
  \item La Hip\'otesis de las Expectativas (HE): plantea que las tasas de inter\'es de largo plazo deben reflejar por completo la informaci\'on revelada por las futuras tasas de inter\'es de corto plazo esperadas (Yung-Shi Liau, Jack J.W. Yang, 2009, p.180), o sea que los tipos de largo plazo no son m\'as que una suma ponderada de los tipos de corto plazo esperados (Fern\'andez J.L., Robles M.D., 2005, p. 244). As\'i, se puede afirmar entonces que la HE es una teor\'ia que plantea que las tasas de inter\'es exclusivamente representan las tasas previstas en el futuro.
\end{itemize}


\hspace*{0.4 cm} Uno de los principales objetivos que se persigue mediante el uso de esta herramienta, es el de estimar los precios de los t\'itulos de la deuda p\'ublica nacional que posee una instituci\'on financiera en su portafolio de inversiones en un per\'iodo determinado. 


\section{Deuda P\'ublica Nacional}

\hspace*{0.4 cm} Deuda es la obligaci\'on que un sujeto tiene de reintegrar, satisfacer o pagar, especialmente dinero. P\'ublico, por otra parte, es un adjetivo que refiere a aquello perteneciente a toda la sociedad o que es com\'un del pueblo.

\hspace*{0.4 cm} La noci\'on de deuda p\'ublica hace menci\'on al conjunto de deudas que mantiene el Estado frente a otro pa\'is o particulares. Se trata de un mecanismo para obtener recursos financieros a trav\'es de la emisi\'on de t\'itulos de valores.

\hspace*{0.4 cm} El Estado, por lo tanto, contrae deuda p\'ublica para solucionar problemas de liquidez (cuando el dinero en caja no resulta suficiente para afrontar los pagos inmediatos) o para financiar proyectos a medio o largo plazo.

\hspace*{0.4 cm} La deuda p\'ublica puede ser contra\'ida por la administraci\'on municipal, provincial o nacional. Al emitir t\'itulos de valores y colocarlos en los mercados nacionales o extranjeros, el Estado promete un futuro pago con intereses según los plazos estipulados por el bono.

\hspace*{0.4 cm} La emisi\'on de deuda p\'ublica, al igual que la creaci\'on de dinero y los impuestos, son medios que tiene el Estado para financiar sus actividades. La deuda p\'ublica, de todos modos, tambi\'en puede utilizarse como un instrumento de la pol\'itica econ\'omica, de acuerdo a la estrategia escogida por las autoridades.

\hspace*{0.4 cm} Tendr\'iamos que hablar, por un lado, de tres tipos diferentes de deuda p\'ublica, aunque es cierto que hay diversas clasificaciones. As\'i, aquellos son los siguientes:

\begin{itemize}
  \item A corto plazo. Dentro de esta categor\'ia se encuentran las Letras del Tesoro y se identifica por el hecho de que tiene un plazo de vencimiento que no supera el a\~no.
  \item A medio plazo. Los bonos del Estado son, por su parte, los m\'aximos exponentes de esta clase de deuda p\'ublica que se suele utilizar para hacer frente a lo que ser\'an los gastos ordinarios que tiene aquel.
  \item A largo plazo. Como su propio nombre indica, este tipo de deuda tiene una duraci\'on muy larga, que se fijar\'a convenientemente, y que puede incluso llegar a ser perpetua. En su caso, se recurre a aquel para hacer frente a lo que ser\'ian gastos extraordinarios o para situaciones especiales.
\end{itemize}


\hspace*{0.4 cm} Es posible clasificar la deuda p\'ublica de distintas maneras. La deuda p\'ublica real es aquella compuesta por los t\'itulos que pueden ser adquiridos por los particulares, los bancos privados y el sector exterior. La deuda p\'ublica ficticia, en cambio, es la emisi\'on destinada al Banco Central del pa\'is, que es un organismo de la misma administraci\'on p\'ublica.





\hspace*{0.4 cm} Para ello es importante conocer las caracter\'isticas de los t\'itulos \'o instrumentos que existen en el mercado venezolano, entre ellos tenemos los siguientes,

\vspace{0.4cm}

\begin{itemize}
  \item T\'itulos de inter\'es fijo (TIF): son instrumentos que se caracterizan por tener una renta fija, y se emiten en moneda nacional.
  \item T\'itulos de inter\'es variable (VEBONO): se caracterizan por poseer una renta variable, e igual que los TIF se emiten en Bol\'ivares.
  \item Bonos PDVSA : son instrumentos emitidos en d\'olares.
\end{itemize}

\vspace{0.5cm}

\noindent cabe destacar que en el presente trabajo s\'olo  se considerar\'an aquellos instrumentos emitidos en Bol\'ivares.

\vspace{0.5cm}

\hspace*{0.4 cm}Asociado a cada t\'itulo hay un monto de dinero que se invierte, el cual es entregado al ente emisor por el t\'itulo en s\'i, existe tambi\'en una fecha de emisi\'on, y una fecha de vencimiento, d\'ia en el cual el ente emisor retorna el monto invertido inicialmente. Es importante destacar que estos t\'itulos pagan un inter\'es al portador cada tres meses, y esta representa la ganancia que tiene el inversionista.

\vspace{0.5cm}

\hspace*{0.4 cm} Con el fin de determinar en que t\'itulo es mejor invertir, es necesario conocer el rendimiento al vencimiento que posee dicho t\'itulo, este valor nos da un idea de cu\'al ser\'a el retorno que recibir\'a el portador del t\'itulo por la tenencia o compra del mismo. Para calcular este valor es necesario conocer la fecha de vencimiento del t\'itulo as\'i como su valor nominal y su precio.


 \hspace*{0.4 cm} A partir de la siguiente f\'ormula podemos hallar dicho valor,

\vspace{0.5cm}

\begin{center}

$\displaystyle{P_{t,m} = \sum_{i=1}^{m-1}{\frac{c}{(1+R_{t,i})^i} + \frac{1+c}{(1+R_{t,m})^m}} }$

\end{center}

\vspace{0.5cm}

\noindent donde $P_{t,m}$ representa el precio del t\'itulo al tiempo t, y que tiene un vencimiento de m a\~nos, c representa el cup\'on asociado al t\'itulo, el \'indice $i = 1,...,m$ representa cuantos cupones todav\'ia le quedan al t\'itulo por pagar antes de su vencimiento. Por su parte $R_{t,m}$ representa el rendimiento al vencimiento del t\'itulo en el tiempo t y que tiene un vencimiento de m a\~nos.

\vspace{0.5cm}

\hspace*{0.4 cm}A partir de la f\'ormula anterior podemos afirmar que para calcular el rendimiento al vencimiento de un t\'itulo es necesario conocer su precio es una fecha espec\'ifica, pero esto no siempre es posible, esto se debe a las caracter\'isticas del mercado venezolano ya que son pocos los t\'itulos que cotizan y por ende no se conoce su precio. Dicho precio puede conocerse a diario mediante la informaci\'on suministrada en la pesta\~na ``0-22" del documento ``resumersec" del Banco Central de Venezuela, este ente publica cada d\'ia las operaciones realizadas con estos instrumentos, en este documento se puede obtener el precio de aquellos t\'itulos que coticen ese d\'ia, el problema radica en conocer los precios de aquellos t\'itulos que no est\'en en dicho documento.



\vspace{0.5cm}

\hspace*{0.4 cm} La curva de rendimientos presenta emp\'iricamente una serie de dificultades, debido a que se construye a trav\'es de una serie de precios (tasas) de instrumentos financieros discontinuos en el tiempo que, por lo general, est\'an lejos de ser una curva suave. Para su estimaci\'on existen diversas metodolog\'ias, las param\'etricas y las no param\'etricas. Las metodolog\'ias param\'etricas se basan en modelos asociados a una familia funcional que obedece al comportamiento de alguna distribuci\'on de probabilidad, sobre la cual suponemos que las caracter\'isticas de la poblaci\'on de inter\'es pueden ser descritas. Es as\'i como, los modelos dise\~nados en este contexto, basados en regresi\'on, buscan describir el comportamiento de una variable de inter\'es con otras llamadas ex\'ogenas, a trav\'es de funciones de v\'inculo lineales o no lineales.


\section{Metodolog\'ias Param\'etricas}

\hspace*{0.4 cm} Estad\'isticamente, un modelo param\'etrico es una familia funcional que
obedece al comportamiento de alguna distribuci\'on de probabilidad,sobre la cual suponemos que las caracter\'isticas de la poblaci\'on de inter\'es
pueden ser descritas. Es as\'i como, los modelos dise\~nados en este contexto,
basados en regresi\'on, buscan describir el comportamiento de una
variable de inter\'es con otras llamadas ex\'ogenas, a trav\'es de funciones de
v\'inculo lineales o no lineales.

La curva de Nelson-Siegel

\hspace*{0.4 cm} Nelson y Siegel (1987) introducen un modelo param\'etrico para el ajuste
de los rendimientos hasta la madurez de los bonos del tesoro de Estados
Unidos que se caracteriza por ser parsimonioso y flexible en modelar
cualquier forma t\'ipica asociada con las curvas de rendimientos. La estructura
param\'etrica asociada a este modelo permite analizar el comportamiento
a corto y a largo plazo de los rendimientos y ajustar -sin
esfuerzos adicionales-, curvas mon\'otonas, unimodales o del tipo S.


\hspace*{0.4 cm} Una clase de funciones que genera f\'acilmente las formas usuales de las
curvas de rendimientos es la asociada con la soluci\'on de ecuaciones en
diferencia. La teor\'ia de expectativas sobre la estructura de las tasas de
inter\'es promueve la investigaci\'on en este sentido, dado que si las tasas
spot son producidas por medio de una ecuaci\'on diferencial, entonces las
tasas forward -siendo pron\'osticos-, ser\'an la soluci\'on de las ecuaciones
diferenciales. La expresi\'on param\'etrica propuesta por Nelson y Siegel
(1987) que describe las tasas forward es exhibida a continuaci\'on:


\begin{center}
$\displaystyle{f(m) = \beta_{0} + \beta_{1} e^{\frac{-m}{\tau}} +\beta_{2} \left(\frac{-m}{\tau}\right)e^{\frac{-m}{\tau}}}$
\end{center}

\noindent donde $m$ denota la madurez del activo y $\beta_{0}$, $\beta_{1}$, $\beta_{2}$ y $\tau$ los par\'ametros a ser
estimados. Puesto que las tasas spot pueden ser obtenidas a trav�s de tasas
forward por medio de la expresi\'on:

\begin{center}
$\displaystyle{s(m) = \int_{0}^{m}f(x)dx}$
\end{center}

\noindent la ecuaci\'on que determina las tasas spot $s(m)$ de activos con madurez m es dada por:

\begin{center}
$\displaystyle{s(m) = \beta_{0}+ \beta_{1}\frac{\left(1-e^\frac{-m}{\tau}\right)}{m/\tau} + \beta_{2} \left(\frac{\left(1-e^\frac{-m}{\tau}\right)}{m/\tau} -  e^\frac{-m}{\tau}\right)}$
\end{center}


\noindent cuya ecuaci\'on es lineal si conocemos $\tau$ .

\hspace*{0.4 cm} El valor l\'imite del rendimiento es $\beta_{0}$ cuando el plazo al vencimiento m es grande, mientras que, cuando el plazo al vencimiento m es peque\~no el
rendimiento en el l\'imite es $\beta_{0}+\beta_{1}$. Igualmente, los coeficientes del
modelo de tasas forward pueden ser interpretados como medidas de
fortaleza al corto, mediano y largo plazo. La contribuci\'on al largo plazo
es determinada por $\beta_{0}$, $\beta_{1}$ lo hace al corto plazo ponderado por la
funci\'on mon\'otona creciente (decreciente) $e^{\frac{-m}{\tau}}$ cuando $\beta_{1}$ es negativo
(positivo) y $\beta_{2}$ lo hace al mediano plazo ponderado por la funci\'on
mon\'otona creciente (decreciente) $(\frac{-m}{\tau}) e^{\frac{-m}{\tau}}$ cuando $\beta_{2}$ es negativo
(positivo). Una de las principales utilidades de la curva ha sido para
prop\'ositos de control de la pol\'itica monetaria.

\hspace*{0.4 cm} Consecuentemente, $s(m)$ ser\'a la ecuaci\'on utilizada para captar la relaci\'on
subyacente entre los rendimientos y los plazos al vencimiento o madurez,
sin recurrir a modelos m\'as complejos que involucren un mayor n\'umero
de par\'ametros. Adicionalmente, dado que la curva de Nelson-Siegel
proporciona tasas spot compuestas continuas, estas deben transformarse
en cantidades discretas, a trav\'es de la funci\'on de descuento.


\begin{center}
$\displaystyle{s_{d}(m) = e^{\frac{s(m)}{100}} - 1}$
\end{center}

La curva de Svensson

\hspace*{0.4 cm} En la curva de Nelson-Siegel se destaca que cada coeficiente del modelo
contribuye en el comportamiento de las tasas forward en el corto,
mediano y largo plazo; no obstante, Svensson (1994) propone una nueva
versi\'on de la curva de Nelson-Siegel donde un cuarto t\'ermino es incluido
para producir un efecto adicional y semejante al proporcionado por ??2:
$\beta_{3}(\frac{m}{\tau_{2}})e^{\frac{-m}{\tau_{2}}}$.

\hspace*{0.4 cm} En este caso, la funci\'on para describir la din\'amica de las tasas forward es,

\begin{center}
$\displaystyle{f(m) = \beta_{0} + \beta_{1} e^{\frac{-m}{\tau_{1}}} +\beta_{2} \left(\frac{-m}{\tau_{1}}\right)e^{\frac{-m}{\tau_{1}}} + \beta_{3}\left(\frac{-m}{\tau_{2}}\right)e^{\frac{-m}{\tau_{2}}} }$
\end{center}

\hspace*{0.4 cm} La curva spot de Svensson puede ser derivada a partir de la curva
forward en forma semejante a la descrita para el modelo de Nelson-
Siegel, obteniendo la siguiente expresi\'on:


\begin{center}
$\displaystyle{s(m) = \beta_{0}+ \beta_{1}\frac{\left(1-e^\frac{-m}{\tau_{1}}\right)}{m/\tau_{1}} + \beta_{2} \left(\frac{\left(1-e^\frac{-m}{\tau_{1}}\right)}{m/\tau_{1}} -  e^\frac{-m}{\tau_{1}}\right) + \beta_{3} \left(\frac{\left(1-e^\frac{-m}{\tau_{2}}\right)}{m/\tau_{2}} -  e^\frac{-m}{\tau_{2}}\right)}$
\end{center}

\hspace*{0.4 cm} La funci\'on de descuento tiene que ser utilizada con el fin de obtener las
tasas estimadas para cada d\'ia de negociaci\'on o trading. Svensson (1994)
propone estimar los par\'ametros de la curva cero cup\'on (curva spot),
minimizando una medida de ajuste tal como la suma de cuadrados del
error sobre los precios spot; sin embargo, enfatiza en que los precios
pueden llegar a ser mal ajustados para los activos de madurez corta. En
lugar de llevar el an\'alisis por este camino, propone estimar los
rendimientos fundamentado, principalmente, en que las decisiones de la
pol�tica econ\'omica se basan en el comportamiento de las tasas y que
obteniendo las tasas a trav\'es de la curva, los precios pueden ser
calculados una vez la funci\'on de descuento es evaluada. De esta manera,
los par\'ametros son escogidos minimizando la suma de cuadrados de la
diferencia entre los rendimientos observados y estimados por la curva.

\hspace*{0.4 cm} La estimaci\'on es realizada por medio de m\'axima verosimilitud, m\'inimos
cuadrados no lineales o el m\'etodo de momentos generalizados. En
muchos casos, como afirma Svensson (1994), el modelo de Nelson-
Siegel proporciona ajustes satisfactorios, aunque en algunos casos
cuando la estructura de las tasas de inter\'es es m�s compleja, el ajuste del
modelo de Nelson-Siegel es poco satisfactorio y el modelo de Svensson
logra desempe\~narse mejor.


Polinomios de componentes principales

\hspace*{0.4 cm} Hunt y Terry (1998) propone un ajuste de la curva de rendimientos
utilizando polinomios. Si frecuentemente la curva es especificada como,

\begin{center}
$\displaystyle{y(\tau) = \beta_{0} + \beta_{1}\tau +\beta_{2}\tau^2 +\beta_{3}\tau^3}$
\end{center}


\hspace*{0.4 cm} La cual puede captar todas la formas que puede asumir la curva, su
principal problema recae en el ajuste para aquellas tasas con per\'iodos de
vencimiento bastante largos. Aunque los autores conocen sobre las
propiedades de parsimonia y de ajuste asociados con la curva de Nelson-
Siegel, critican los problemas que acarrea la estimaci\'on de sus
par\'ametros, proponiendo el ajuste de la curva de polinomios, bajo
algunas modificaciones.

\hspace*{0.4 cm} Una transformaci\'on sobre el t\'ermino de plazos ($\tau$) que remueve la
inestabilidad asociada con las tasas a largo plazo del polinomio (5) es
sugerida. El modelo recomendado, siguiendo la notaci\'on de Hunt y
Terry (1998) es:

\begin{center}
$\displaystyle{y(\tau) = \beta_{0} + \sum_{i=1}^{p} \beta_{i} \frac{1}{(1+\tau)^i}} $
\end{center} 

\noindent donde

\begin{center}
$\displaystyle{y(0) = \sum_{i=0}^{p}\beta_{i} \hspace*{0.2 cm} y \hspace*{0.2 cm} y(\infty) = \beta_{0}   }$
\end{center} 


\hspace*{0.4 cm}Investigaciones relacionadas con curvas de rendimientos, han llegado a
la conclusi\'on que modelos con tres o cuatro par\'ametros son suficientes
para obtener un buen ajuste de los datos (Hunt 1995). Por tal motivo,
Hunt y Terry (1998) proponen restringir p a tres o cuatro. Aunque este
n\'umero de par�metros no necesariamente determina si realmente la
bondad de ajuste pueda llegar a ser satisfactoria, los autores proponen
utilizar componentes principales sobre los primeros p t\'erminos
polinomiales $1/(1 + \tau)$, con el fin de seleccionar $k<p$ variables, a ser
incluidas en la ecuaci\'on (6). Utilizar las componentes principales
proporcionar\'a un menor error de ajuste en comparaci\'on con (5),
debido a su capacidad para captar variabilidad. Una descripci\'on
detallada respecto al c\'alculo de las componentes principales en el
esquema polinomial es dada por Hunt y Terry (1998).


Polinomios trigonom\'etricos

\hspace*{0.4 cm} Las funciones trigonom\'etricas pueden ser utilizadas para capturar de
forma satisfactoria las distintas configuraciones que pueden asumir las
curvas de rendimientos. En este caso, el modelo puede ser descrito como
$y(\tau) = \beta_{0} + \beta_{1}cos(\gamma_{1}\tau) + \beta_{2}sen(\gamma_{2}\tau)$; donde ?? representa la duraci\'on o la
madurez del papel, en tanto que $\beta_{0}$, $\beta_{1}$, $\beta_{2}$, $\gamma_{1}$ y $\gamma_{2}$ son los par\'ametros
objeto de inter\'es. Cualquier metodolog\'ia de optimizaci\'on no lineal puede
ser utilizada para estimar los par\'ametros del modelo (Nocedal y Wright
1999). Aunque podr\'ia asumirse un par\'ametro de fase en el modelo, este
no es considerado por motivos de parsimonia.

\section{Metodolog\'ias no Param\'etricas}

\hspace*{0.4 cm} La regresi\'on no param\'etrica se ha convertido en los \'ultimos a\~nos en un
\'area de excesivo estudio, debido a sus ventajas relativas respecto a los
modelos de regresi\'on basado en funciones. Entre las caracter\'isticas m\'as
importantes de estos modelos tenemos, la flexibilidad en los supuestos y
el ajuste dirigido espec\'ificamente a trav\'es de los datos.


\hspace*{0.4 cm} Dentro de un marco estad\'istico supondremos que tenemos un conjunto
de n observaciones $(x_{i}, y_{i})$, $i= 1, 2,., n$, independientes, donde se intenta
establecer las relaciones existentes entre una respuesta y un conjunto de
variables explicativas de forma semejante a los modelos de regresi\'on
cl\'asica.


\hspace*{0.4 cm} El modelo que relaciona este conjunto de variables es dado por:

\begin{center}
$\displaystyle{y_{i} = m(x_{i}) + \epsilon_{i}}$
\end{center} 



\noindent donde la funci\'on $m(.)$ no espec\'ifica una relaci\'on param\'etrica, sino
permitir que los datos determinen la relaci\'on funcional apropiada. Bajo
estas condiciones la idea es que la media m(.) sea suave, suavidad que
puede controlarse acotando la segunda derivada, $|m''(x)|???M$, para todo
x y M una constante.

Regresi\'on Kernel

\hspace*{0.4 cm} El m\'etodo m\'as simple de suavizamiento es el suavizador Kernel. Un
punto x se fija en el soporte de la funci\'on $m(.)$ y una ventana de
suavizamiento es definida alrededor de x. Frecuentemente, la ventana de
suavizamiento es simplemente un intervalo de la forma $(x ??? h, x + h)$,
donde h es un par\'ametro conocido como bandwidth.

\hspace*{0.4 cm} La estimaci\'on Kernel es un promedio ponderado de las observaciones
dentro de la ventana de suavizamiento,

\begin{center}
$\displaystyle{\hat{m}(x) = \frac{\sum_{i=1}^{n} K(\frac{x_{i}-x}{h}) y_{i}}{\sum_{i=1}^{n} K(\frac{x_{i}-x}{h})}}$
\end{center}

\noindent donde $K(.)$ es la funci\'on Kernel de ponderaci\'on. La funci\'on Kernel es escogida
de tal forma que las observaciones m\'as pr\'oximas a x reciben mayor peso. Una
funci\'on frecuentemente utilizada es la bicuadr\'atica:

$$ K(x) = \left\{ % para la llave grandota
        \begin{tabular}{cc}
        	$(1-x^2)^2$ & si $-1 \leq x  \leq 1$ \\
        	$0$ & si $x \ge 1, \hspace*{0.2 cm} x<-1$ \\
        \end{tabular}
\right. $$



\hspace*{0.4 cm} Sin embargo, otro tipo de funciones de peso son utilizadas, tal como la
gaussiana, $K(x) = (2 \sqrt{\pi})^{-1} e^{\frac{-x^2}{2}}$ y la familia beta sim�trica $K(x) = \frac{(1-x^2)_{+}^{\gamma}}{Beta(0.5,\gamma+1)}, \hspace*{0.2 cm}\gamma = 0,1,...$















\hspace*{0.4 cm} Por su parte, las metodolog\'ias no param\'etricas se han convertido en los \'ultimos a\~nos en un
\'area de gran estudio debido a sus ventajas relativas respecto a los modelos de regresi\'on basado en funciones. Entre las caracter\'isticas m\'as importantes de estos modelos tenemos, la flexibilidad en los supuestos y el ajuste dirigido espec\'ificamente a trav\'es de los datos. Entre estas metodolog\'ias se destacan la de regresi\'on Kernel, polinomios locales, splines de polinomios (Fan y Gibels, $1996$ [6]), splines c\'ubicos suavizados (B.W. Silverman, $1985$ [7]), super suavizador de Friedmann (Friedmann, $1984$ [8]) y redes neuronales artificiales (Sharda, $1994$ [9]).

\vspace{0.5cm}

\hspace*{0.4 cm} En el siguiente trabajo se propone el uso de la metodolog\'ia de splines c\'ubicos suavizados, la cual posee la ventaja de contar un factor de penalizaci\'on el cu\'al es muy \'util al momento de tener un balance entre la suavidad de la curva y su bondad de ajuste. A grandes rasgos estas metodolog\'ias se basan en estimar la curva de rendimientos de dichos t\'itulos, curva que relaciona el rendimiento al vencimiento con la maduraci\'on o fecha de vencimiento, con el fin de estimar los precios de los t\'itulos a un d\'ia en espec\'ifico. De esta manera a partir de una determinada fecha es posible mediante estas metodolog\'ias estimar el rendimiento al vencimiento de un t\'itulo y por ende saber su precio estimado.

\newpage

\section{Objetivos.}

\subsection{Objetivos  generales del trabajo.}

\begin{itemize}
  \item Estimar la curva de rendimientos mediante el uso de los Splines C\'ubicos Suavizados.
\end{itemize}

\subsection{Objetivos espec\'ificos del trabajo.}

\begin{itemize}
  \item Generar un hist\'orico para los t\'itulos de tasa de inter\'es fija (TIF), pertenecientes a la deuda p\'ublica nacional (DPN).
  \item Estimar la curva de rendimientos para los TIF.
  \item Generar un hist\'orico para los t\'itulos de tasa de inter\'es variable (VEBONO), pertenecientes a la deuda p\'ublica nacional (DPN).
  \item Estimar la curva de rendimientos para los VEBONO.
  \item Estimar los precios de los TIF pertenecientes a un portafolio en un momento espec\'ifico.
  \item Estimar los precios de los VEBONO pertenecientes a un portafolio en un momento espec\'ifico.

\end{itemize}





\chapter{Marco te\'orico.}

\section{Teor\'ia de interpolaci\'on.}

\hspace{0.4cm} En el subcampo matem\'atico del an\'alisis num\'erico, se denomina interpolaci\'on a la obtenci\'on de nuevos puntos partiendo del conocimiento de un conjunto discreto de puntos.


\hspace{0.4cm} En ciertos casos el usuario conoce el valor de una funci\'on $f(x)$ en una serie de puntos $x_{1}, x_{2},..., x_{N}$, pero no se conoce una 
expresi\'on anal\'itica de $f(x)$ que permita calcular el valor de la funci\'on para un punto arbitrario. Un ejemplo claro son las mediciones de laboratorio, donde se mide cada minuto un valor, pero se requiere el valor en otro punto que no ha sido medido. Otro ejemplo son mediciones de temperatura en la superficie
de la Tierra, que se realizan en equipos o estaciones meteorol\'ogicas y se necesita calcular la temperatura en un punto cercano, pero distinto al punto de medida.

\hspace{0.4cm} La idea de la interpolaci\'on es poder estimar $f(x)$ para un valor de x arbitrario, a partir de la construcci\'on de una curva o superficie que une los puntos donde se han realizado las mediciones y cuyo valor si se conoce. Se asume que el punto arbitrario x se encuentra dentro de los l\'imites de los puntos de medici\'on, en caso contrario se llamar\'ia extrapolaci\'on. 

\hspace{0.4cm} Un proceso de interpolaci\'on se realiza en dos etapas:

\begin{itemize}
  \item Hacer un ajuste de los datos disponibles con una funci\'on interpolante.
  \item Evaluar la funci\'on interpolante en el punto de inter\'es x.
\end{itemize}


\hspace{0.4cm} Este proceso en dos etapas no es necesariamente el m\'as 
eficiente. La mayor\'ia de algoritmos comienzan con un punto cercano $f(x_{i})$, y poco a poco van aplicando correcciones m\'as pequenas a medida que la 
informaci\'on de valores $f(xi)$ m\'as distantes son incorporadas. El procedimiento toma aproximadamente $O(N^{2})$ operaciones. Si la funci\'on tiene un comportamiento suave, la \'ultima correci\'on ser\'a la m\'as pequena y puede ser utilizada para estimar un l\'imite a rango de error.


\hspace{0.4cm} Dentro de las intepolaciones m\'as usadas podemos destacar,

\begin{itemize}
  \item Interpolaci\'on lineal
  \item Interpolaci\'on de Hermite
  \item Interpolaci\'on polin\'omica 
  \item Interpolaci\'on de Splines 

\end{itemize}

Interpolaci\'on lineal

\hspace{0.4cm} La interpolaci\'on lineal es un procedimiento muy utilizado para estimar los valores que toma una funci\'on en un intervalo del cual conocemos sus valores en los extremos $(x_{1}, f(x_{1}))$ y $(x_{2},f(x_{2}))$. Para estimar este valor utilizamos la aproximaci\'on a la funci\'on f(x) por medio de una recta $r(x)$ (de ah\'i el nombre de interpolaci\'on lineal, ya que tambi\'en existe la interpolaci\'on cuadr\'atica). 


\hspace{0.4cm} La expresi\'on de la interpolaci\'on lineal se obtiene del polinomio interpolador de Newton de grado uno. Aunque s\'olo existe un \'unico polinomio que interpola una serie de puntos, existen diferentes formas de calcularlo. Este m\'etodo es \'util para situaciones que requieran un n\'umero bajo de puntos para interpolar, ya que a medida que crece el n\'umero de puntos, tambi\'en lo hace el grado del polinomio.

\hspace{0.4cm} Existen ciertas ventajas en el uso de este polinomio respecto al polinomio interpolador de Lagrange. Por ejemplo, si fuese necesario a\~nadir alg\'un nuevo punto o nodo a la funci\'on, tan s\'olo habr\'ia que calcular este \'ultimo punto, dada la relaci\'on de recurrencia existente y demostrada anteriormente.

\hspace{0.4cm}El primer paso para hallar la f\'ormula de la interpolaci\'on es definir la pendiente de orden n de manera recursiva, as\'i,

\begin{itemize}
  \item $f_{0}(x_{i})$ es el t\'ermino i-\'esimo de la secuencia. 
  \item $\displaystyle{f_{1}(x_{0},x_{1}) = \frac{f_{0}(x_{1})-f_{0}(x_{0})}{x_{1}-x_{0}}}$
  \item $\displaystyle{f_{2}(x_{0},x_{1},x_{2}) = \frac{f_{1}(x_{1},x_{2})-f_{1}(x_{0},x_{1})}{x_{2}-x_{0}}}$
\end{itemize}

\hspace{0.4cm} En general,

\begin{center}
$\displaystyle{f_{i}(x_{0},x_{1},...,x_{i-1},x_{i}) = \frac{f_{i-1}(x_{1},...,x_{i-1},x_{i})-f_{i-1}(x_{0},x_{1},...,x_{i-1} )}{x_{i}-x_{0}}}$
\end{center}


\noindent donde  $\displaystyle{x_{i}-x_{j}}$ representa la distancia entre dos elementos.

\hspace{0.4cm} Puede apreciarse c\'omo en la definici\'on general se usa la pendiente del paso anterior, $f_{i-1}(x_{1},...,x_{i-1},x_{i})$, a la cual se le resta la pendiente previa de mismo orden, es decir, el sub\'indice de los t\'erminos se decrementa en 1, como si se desplazara, para obtener $f_{i-1}(x_{0},x_{1},...,x_{i-1})$.

\hspace{0.4cm} N\'otese tambi\'en que aunque el t\'ermino inicial siempre es $x_{0}$ este puede ser en realidad cualquier otro, por ejemplo, se puede definir $f_{1}(x_{i-1},x_{i})$ de manera an\'aloga al caso mostrado arriba.

\hspace{0.4cm} Una vez conocemos la pendiente, ya es posible definir el polinomio de grado n de manera tambi\'en recursiva, as\'i,

\begin{itemize}
  \item $p_{0}(x) = f_{0} (x_{0}) =x_{0}$.  Se define as\'i ya que este valor es el \'unico que se ajusta a la secuencia original para el primer t\'ermino. 
  \item $\displaystyle{p_{1}(x) = p_{0}(x) +  f_{1} (x_{0},x_{1}) (x-x_{0})}$.
  \item $\displaystyle{p_{2}(x) = p_{1}(x) +  f_{2} (x_{0},x_{1},x_{2}) (x-x_{0})(x-x_{1})}$.
\end{itemize}

\hspace{0.4cm} En general,

\begin{center}
$\displaystyle{p_{i}(x) = p_{i-1}(x) +  f_{i} (x_{0},x_{1},...,x_{i-1},x_{i}) \prod_{j=0}^{i-1}(x-x_{j})}$
\end{center}


Interpolaci\'on de Lagrange

\hspace{0.4cm} Empezamos con un conjunto de $n+1$ puntos en el plano (que tengan diferentes coordenadas x), $(x_{0}, y_{0}), (x_{1}, y_{1}), (x_{2}, y_{2}),...,(x_{n}, y_{n})$. As\'i, queremos encontrar una funci\'on polin\'omica que pase por esos $n+1$ puntos y que tengan el menor grado posible. Un polinomio que pase por varios puntos determinados se llama un polinomio de interpolaci\'on.

\hspace{0.4cm} Una posible soluci\'on viene dada por el  polinomio de interpolaci\'on de Lagrange. Lagrange public\'o su f\'ormula en 1795 pero ya hab\'ia sido publicada en 1779 por Waring y redescubierta por Euler en 1783.

\hspace{0.4cm} Dado un conjunto de $k + 1$ puntos $(x_{0}.x_{1}),...,(x_{k}.x_{k+1})$, donde todos los $x_{j}$ se asumen distintos, el polinomio interpolador en la forma de Lagrange es la combinaci\'on lineal,

\begin{center}
$\displaystyle{L(x) = \sum_{j=0}^{k} y_{j} l_{j}(x)}$
\end{center}


\noindent donde $ l_{j}(x)$ son las bases polin\'omicas de Lagrange ,

\begin{center}
$\displaystyle{l_{j}(x) = \prod_{i=0,i \neq j}^{k} \frac{x-x_{i}}{x_{j}-x_{i}}  }$
\end{center}

\hspace{0.4cm} La funci\'on que estamos buscando es una funci\'on polin\'omica $L(x)$ de grado $k$ con el problema de interpolaci\'on puede tener tan solo una soluci\'on, pues la diferencia entre dos tales soluciones, ser\'ia otro polinomio de grado $k$ a lo sumo, con $k+1$ ceros. Por lo tanto, $L(x)$ es el \'unico polinomio interpolador.


\hspace{0.4cm} Si se aumenta el n\'umero de puntos a interpolar (o nodos) con la intenci\'on de mejorar la aproximaci\'on a una funci\'on, tambi\'en lo hace el grado del polinomio interpolador as\'i obtenido, por norma general. De este modo, aumenta la dificultad en el c\'alculo, haci\'endolo poco operativo manualmente a partir del grado 4, dado que no existen m\'etodos directos de resoluci\'on de ecuaciones de grado 4, salvo que se puedan tratar como ecuaciones bicuadradas, situaci\'on extremadamente rara.

\hspace{0.4cm} La tecnolog\'ia actual permite manejar polinomios de grados superiores sin grandes problemas, a costa de un elevado consumo de tiempo de computaci\'on. Pero, a medida que crece el grado, mayores son las oscilaciones entre puntos consecutivos o nodos. Se podr\'ia decir que a partir del grado 6 las oscilaciones son tal que el m\'etodo deja de ser v\'alido, aunque no para todos los casos.

\hspace{0.4cm} Sin embargo, pocos estudios requieren la interpolaci\'on de tan s\'olo 6 puntos. Se suelen contar por decenas e incluso centenas. En estos casos, el grado de este polimonio ser\'ia tan alto que resultar\'ia inoperable. Por lo tanto, en estos casos, se recurre a otra t\'ecnica de interpolaci\'on, como por ejemplo a la Interpolaci\'on polin\'omica de Hermite o a los splines c\'ubicos. Otra gran desventaja, respecto a otros m\'etodos de interpolaci\'on, es la necesidad de recalcular todo el polinomio si se var\'ia el n\'umero de nodos.

\hspace{0.4cm} Aunque el polinomio interpolador de Lagrange se emplea mayormente para interpolar funciones e implementar esto f\'acilmente en una computadora, tambi\'en tiene otras aplicaciones en el campo del \'algebra exacta, lo que ha hecho m\'as c\'elebre a este polinomio, por ejemplo en el campo de los proyectores ortogonales.


Interpolaci\'on de Hermite

\hspace{0.4cm} El objetivo de esta interpolaci\'on Hermite es minimizar el error producido en la interpolaci\'on de Lagrange de la funci\'on $f(x)$ sobre el intervalo $[a, b]$ sin aumentar el grado del polinomio interpolador.


\hspace{0.4cm} Dados un entero no negativo N y  N + 1 puntos $(x_{0},..., x_{N})$ de la recta distintos dos a dos y los valores $f^{(j)} (x_{i})$, donde $0<i<N$ y $0<j<k_{i-1}$ de una funci\'on f y de sus derivadas, encontrar un polinomio de grado $m = (k_{0} +k_{1} +...+k_{n-1},)$ tal que,

\begin{center}
$\displaystyle{P^{(j)} (x_{i}) = f^{(j)} (x_{i}), para \hspace{0.2cm} 0<i<N \hspace{0.2cm}y \hspace{0.2cm} 0<j<k_{i-1}}$
\end{center}

\hspace{0.4cm} El problema de interpolaci\'on de Hermite tiene soluci\'on \'unica, que se llama polinomio interpolador de Hermite. En lugar de interpolar sobre un soporte de puntos (de Tchebycheff) donde en general se desconoce el valor de la funci\'on, de hace de otra manera, imponiendo unas condiciones al polinomio,

\begin{itemize}
  \item  Igualar el valor de la funci\'on en en los puntos del soporte, $P (x_{i}) = f^{(j)}$
  \item Igualar el valor de algunas derivadas de la funci\'on tambi\'en en los puntos del soporte, $P^{(j)} (x_{i}) = f^{(j)} (x_{i})$
\end{itemize}

\hspace{0.4cm} Por lo que podemos dejar el polinomio de Hermite de grado (n-1) expresado de la siguiente manera,

\begin{center}
$\displaystyle{L(x) = \sum_{j=0}^{k} y_{j} l_{j} (x) }$
\end{center}

\noindent donde $l_{j}(x)$ son los polinomios de la base de Lagrange.


Interpolaci\'on Polin\'omica

\hspace{0.4cm}Asumamos que se tiene una tabla con n puntos, $(x_{1},y_{1})...,(x_{n},y_{n})$, donde los valores $x_{i}$, para $i=1,...,n$ est\'an ordenados de forma creciente y todos ellos son distintos. Supongamos que dichos puntos se representan en un plano cartesiano y se quiere determinar una curva suave que interpole dichos valores. As\'i, se desea determinar una curva que est\'e definida para todos los $x$ y tome los valores correspondientes de $y$, esto es, que interpole todos los datos de la tabla. Cabe destacar que los puntos considerados se les conoce como nodos.


\vspace{0.5cm}

\hspace{0.4cm} La primera idea natural es usar una funci\'on polinomial que represente esta curva, la cual se puede representar como sigue,

\vspace{0.5cm}
\begin{center}

$\displaystyle{P_{n} = \sum_{i=0}^{n-1} a_{i}x^{i}}$

\end{center}

\noindent tal curva se le conoce como funci\'on polinomial interpoladora de grado n. N\'otese que en cada nodo se satisface que $P_{n}(x_{k})=y_{k}$, donde $k=1,...,n$.

\vspace{0.5cm}

\hspace{0.4cm} As\'i, se tiene que si la tabla es representada mediante una funci\'on subyacente $f(x)$ tal que $f(x_{k})=y_{k}$, para todo k, entonces esta funci\'on puede ser aproximada mediante $P_{n}$, en los puntos intermedios.


\vspace{0.5cm}

\hspace{0.4cm} No ser\'ia descabellado pensar que a medida que los nodos se incrementan, la aproximaci\'on ser\'ia cada vez mejor, lamentablemente esto no siempre es cierto, debido a que en el caso de tener una data con mucho ruido la interpolaci\'on no tendr\'ia mucho sentido ya que la varianza de los valores interpolados ser\'ia muy grande. En este caso, los polinomios interpoladores resultantes ser\'ian una mala representaci\'on de la funci\'on subyacente.

\vspace{0.5cm}

\hspace{0.4cm} Para evitar este fen\'omeno, puede ser de utilidad relajar la condici\'on de que $f(x)$ deber\'ia ser una funci\'on que interpole todos los valores dados y en su lugar usar un trozo de un polinomio local de interpolaci\'on. La funci\'on mediante la cual se logra esta interpolaci\'on se le conoce como spline.


\section{Splines.}

\vspace{1 cm}

\hspace{0.4cm} Una funci\'on spline $S(x)$ es una funci\'on que consta de trozos de polinomios unidos por ciertas condiciones de suavizado. Un ejemplo simple, es una funci\'on poligonal (spline de primer grado), la cual se forma por polinomios lineales unidos, los cuales se definen entre cada par de nodos. Entre los nodos $x_{j}$ y $x_{j+1}$ se define un spline de primer grado como sigue, \\

\begin{center}

$\displaystyle{S(x) = a_{j}x + b_{j} = S_{j}(x)}$

\end{center}

\vspace{0.5cm}

\noindent este spline es lineal. Usualmente $S(x)$ se define como $S_{1}(x)$ para $x<x_{1}$ y como $S_{n-1}(x)$ para $x>x_{n}$, donde $x_{1}$ y $x_{n}$ son nodos frontera.


\begin{figure}[h]
  \scalebox{0.50}{\includegraphics{images/spline_lineal.png}}
\caption{Spline Lineal.}
\end{figure}




\vspace{0.5cm}

\hspace{0.4cm} Un spline de segundo grado es una uni\'on de polinomios cuadr\'aticos tal que $S(x)$ y su derivada $S^{(1)}(x)$ son continuas. Por su parte un spline c\'ubico, se representa mediante la uni\'on de polinomios c\'ubicos con primera y segunda derivada continuas. Este spline debido a su flexibilidad es el m\'as usado en las aplicaciones.

\vspace{0.5cm}

\begin{figure}[h]
  \scalebox{0.50}{\includegraphics{images/spline_cuadratico.png}}
\caption{Spline Cuadr\'atico.}
\end{figure}


\hspace{0.4cm}Formalmente un spline c\'ubico con nodos $x_{1},...x_{n}$ se define a partir de un conjunto de polinomios de la forma,\\

\begin{center}

$\displaystyle{S_{j}(x) = a_{j} + b_{j}x +c_{j}x^2 +d_{j}x^3}$
\end{center}


\vspace{0.5cm}

\noindent con $x_{j}<x<x_{j+1}$, sujeto a las siguientes condiciones,\\


\begin{center}

$\displaystyle{a_{j-1} + b_{j-1}x_{j} +c_{j-1}x_{j}^2 +d_{j-1}x_{j}^3 = a_{j} + b_{j}x_{j} +c_{j}x_{j}^2 +d_{j}x_{j}^3}$\\
$\displaystyle{ b_{j-1} +2c_{j-1}x_{j} +3d_{j-1}x_{j}^2 = b_{j} +2c_{j}x_{j} +3d_{j}x_{j}^2}$\\
$\displaystyle{ 2c_{j-1} +6d_{j-1}x_{j} = 2c_{j} +6d_{j}x_{j}}$\\
$\displaystyle{ c_{0} = d_{0} = c_{n} =d_{n}}$

\end{center}

\begin{figure}[h]
  \scalebox{0.50}{\includegraphics{images/spline_cubico.png}}
\caption{Spline C\'ubico.}
\end{figure}


\hspace{0.4cm}As\'i para n nodos, existen $4(n-1)$ variables y $4(n-1)-2$ restricciones. Las mismas se deben a la necesidad de que el spline c\'ubico sea igual en los valores dados en cada nodo. Las primeras tres restricciones aseguran que la funci\'on resultante en su primera y segunda derivada sean continuas en los nodos. La restricci\'on final significa que el spline c\'ubico es lineal en el punto inicial y final de la muestra. Sin embargo, es importante resaltar que el spline c\'ubico tiene tercera derivada discontinua en los nodos.

\hspace{0.4cm}Debido a que hacen falta dos restricciones de borde, estas se deben a\~nadir. As\'i  $S^{(2)}(x_{1}) = S^{(2)}(x_{n}) = 0$ son las restricciones faltantes, estan hacen referencia a que el spline sea un spline c\'ubico natural. Como se mencion\'o al inicio si se considera una interpolaci\'on polinomial global de un conjunto de datos con mucho ruido pueden surgir aproximaciones no deseables e inestables. En constrate, un spline c\'ubico de interpolaci\'on encaja perfectamente con la suavidad de la funci\'on subyacente.

\begin{figure}[h]
  \scalebox{0.50}{\includegraphics{images/Comparativo_splines.png}}
\caption{Comparativo Splines.}
\end{figure}

\vspace{0.5cm}

\hspace{0.4cm} Otra caracter\'istica de los splines es que con la adici\'on de un par\'ametro s\'olo se aumenta la dimensionalidad del espacio de par\'ametros en una unidad, ya que tres de los cuatro par\'ametros est\'an restringidos. De igual forma, al incrementar el n\'umero de nodos los splines toman formas funcionales m\'as flexibles, lo cual muestra la relaci\'on entre el grado aproximaci\'on que se logra con el spline y el n\'umero de nodos que lo definen.

\vspace{0.5cm}

\hspace{0.4cm} Mientras que las funciones spline son una herramienta interesante para interpolar funciones suaves, encontrarlas num\'ericamente no es tarea f\'acil. Una manera eficiente y muy estable para generar los splines necesarios para aproximar la funci\'on subyacente $f(x)$, es usando las bases de los B-splines c\'ubicos.

\vspace{0.5cm}

\hspace{0.4cm} Supongamos que tenemos un conjunto infinito de nodos $...<x_{-2}<x_{-1}<x_{0}<x_{1}<x_{2}<...$, entonces el j-\'esimo B-spline de grado cero es igual a $B^{0}_{j}(x)=1$, si $x_{j} \leq x \leq x_{j+1}$ y $B^{0}_{j}(x)=0$ en otro caso. Con la funci\'on $B^{0}_{j}(x)$ como punto de partida se puede generar B-splines de grados mayores mediante la siguiente f\'ormula recursiva,\\

\begin{center}

$\displaystyle{B^{k}_{j}(x) = \frac{(x-x_{j})B^{k-1}_{j}(x)}{x_{j+k}-x_{j}} + \frac{(x_{j+k+1}-x)B^{k-1}_{j+1}(x)}{x_{j+k+1}-x_{j+1}}}$
\end{center}

\vspace{0.5cm}

\noindent para $k\geq 1$. As\'i un B-spline de grado $k$ se define como,\\

\begin{center}

$\displaystyle{S^{k}(x) = \sum_{j=-\infty}^{\infty} \theta^{k}_{j} B^{k}_{j-k}(x)}$
\end{center}

\vspace{0.5cm}

\hspace{0.4cm} Una buena interrogante ser\'ia el como se determina los coeficientes $\theta^{k}_{j}$ en la expresi\'on anterior. Note que los B-splines de grado positivo no son ortogonales y por ende no poseen una expresi\'on simple para sus coeficientes.


Sin embargo, los c\'alculos empleados para los B-splines interpoladores de grado cero y uno, son bastante sencillos,\\

\begin{center}

$\displaystyle{S^{0}(x) = \sum_{j=-\infty}^{\infty} y_{j} B^{0}_{j}(x),\hspace{0.4cm} S^{1}(x) = \sum_{j=-\infty}^{\infty} y_{j} B^{1}_{j-1}(x) }$
\end{center}

\vspace{0.5cm}

\hspace{0.4cm}Para splines de grados m\'as elevados, algunas arbitrariedades surgen al momento de calcular estos coeficientes. Por lo tanto, debido a que en las aplicaciones estadisticas existe un mayor inter\'es por encontrar una aproximaci\'on que una interpolaci\'on, la t\'ecnica de minimos cuadros puede ser empleada para calcular estos valores.


\vspace{0.5cm}

\hspace{0.4cm} Ahora bien, supongamos que se tiene un conjunto de $m$ funciones diferenciables $f(x)$, con soporte en el intervalo $[a,b]$, las cuales satifacen las siguientes condiciones,

% begin{itemize}
%   \item $f(x_{i})=y_{i}$, para i=1...,n
%   \item La m-1 derivada $f^{(m-1)}(x)$, es continua en x.
% \end{itemize}

\begin{itemize}
  \item[(i)] $f(x_{i})=y_{i}$, para $i=1...,n$.
  \item[(ii)] La m-1 derivada $f^{(m-1)}(x)$, es continua en x.
\end{itemize}

\hspace{0.4cm} El problema es encontrar entre todas esas funciones, una funci\'on tal que tenga la m\'inima integral del cuadro de su segunda derivada, esto es, una funci\'on que tenga el valor m\'as peque\~no de $\int_{a}^{b} (f^{(m)}(x))^2 dx$. Dicha funci\'on ser\'a la elecci\'on m\'as \'optima al momento de hallar un balance entre suavidad y ajuste de los datos.

\vspace{0.5cm}

\hspace{0.4cm} Se puede desmostrar que la soluci\'on de este problema es \'unica y la funci\'on en cuesti\'on es un spline polinomial que cumple la condici\'on i), y adem\'as satisface que,

\begin{itemize}
  \item[(a)] f es un pilinomio de grado no mayor que $m-1$ cuando $x \in [a,x_{1}]$ y $x \in [x_{n},b]$ .
  \item[(b)] F es un polinomio de grado no mayor a $2m-1$ para puntos interiores, $x \in [x_{i},x_{i+1}]$ con i=1,...,n.
  \item[(c)] f(x) tiene $2m-2$ derivadas continuas en el eje real.
\end{itemize}

\hspace{0.4cm} En resumen, la funci\'on $f$ m\'inima es un spline el cual est\'a conformado por trozos de polinomios unidos en los nodos $x_{i}$, donde dicha funci\'on tiene $2m-2$ derivadas continuas. N\'otese que en muchas aplicaciones $m=2$ es un valor muy utilizado y cuya soluci\'on viene dada mediante el spline c\'ubico natural.


\section{Regresi\'on no param\'etrica mediante splines de suavizado.}

\hspace{0.4cm} Consideremos el siguiente modelo de regresi\'on homoced\'astico,\\

\begin{center}

$\displaystyle{Y_{i}=f(X_{i})+\epsilon_{i}}, \hspace{0.3cm} para \hspace{0.2cm} i=1,...,n$
\end{center}

\vspace{0.5cm}

\noindent donde los $\epsilon_{i}$ son errores de media cero independientes e id\'enticamente distribuidos.

\vspace{0.5cm}

\hspace{0.4cm} Uno de los posibles m\'etodos para emplear splines es aproximar la funci\'on de regresi\'on subyacente mediante las bases de splines, por ejemplo, la base de los B-splines c\'ubicos. As\'i, se escoge una secuencia fija de nodos $-\infty<t_{1}<t_{2}<...<t_{J}<\infty$, los cuales pueden diferir de los predictores. Luego, se calculan los elementos de la base c\'ubica de spline correspondiente.

\vspace{0.5cm}

\hspace{0.4cm}Es posible mostrar que s\'olo son necesarios $J+4$ elementos de esta base. Denotemos a estos elementos por $B_{j}(x)$, as\'i el spline polinomial lo podemos expresar como sigue,

\begin{center}

$\displaystyle{S(x)=\sum_{j=1}^{J+4} \theta_{j}B_{j}(x)}$
\end{center}

\vspace{0.5cm}

\hspace{0.4cm}Entonces los coeficientes $\theta_{j}$ pueden ser calculados al ser considerados como los par\'ametros que se obtienen al minimizar la suma de los errores al cuadrado,

 \begin{center}

$\displaystyle{\sum_{i=1}^{n} \left[ Y_{i} - \sum_{j=1}^{J+4} \theta_{j}B_{j}(X_{j})\right]^2}$
\end{center}

\vspace{0.5cm}

\hspace{0.4cm}Denotamos por $\hat{\theta_{j}}$ al estimador de m\'inimos cuadrados y definimos el estimador del spline polinomial como sigue,

 \begin{center}

$\displaystyle{ \hat{f}_{n}(x) = \sum_{j=1}^{J+4} \hat{\theta_{j}}B_{j}(x)}$
\end{center}

\vspace{0.5cm} Otro enfoque, se basa en la idea de encontrar un curva suave que minimize la suma penalizada de errores al cuadrado, es decir, que minimize la siguiente expresi\'on,

\begin{equation}\label{min}
  n^{-1}\sum_{j=1}^{n}(Y_{j}-f(X_{j}))^2+\mu \int_{a}^{b} [f^{(m)} (x)]^2 dx
\end{equation}

\vspace{0.5cm}


\noindent para alg\'un $\mu > 0$. As\'i como el enfoque de interpolaci\'on anterior, la soluci\'on de este problema de minimizaci\'on es un spline, el cual recibe el nombre de estimador de spline de suavizado.

\vspace{0.5cm}

\hspace{0.4cm} En particular, para el caso $m=2$ el minimizador de (\ref{min}), es un spline c\'ubico natural. Note que $\mu$ juega el papel de par\'ametro de suavizado, este t\'ermino se puede interpretar como una penalizaci\'on por rugosidad de la funci\'on. Curvas que cambian lenta o suavemente presentan un valor peque\~no de la integral, por ejemplo, en una funci\'on lineal la integral toma el valor de cero.

\vspace{0.5cm}

\hspace{0.4cm} De hecho, la primera suma en (\ref{min}) penaliza la falta de fidelidad de la aproximaci\'on de la data mediante el spline. El segundo t\'ermino es el responsable de la suavidad de la aproximaci\'on obtenida mediante el spline. Para ver esto consid\'erese los casos extremos, es decir, cuando $\mu =0$ y $\mu=\infty$. El primer caso conduce a una interpolaci\'on, esto es $\hat{f}(X_{i})=Y_{i}$ para $i=1,...,n$. El otro caso, conduce a una regresi\'on lineal pues $f^{(2)}(x)\equiv 0$.

\vspace{0.5cm}

\hspace{0.4cm} Por lo tanto $\mu$ es el par\'ametro de suavizado que controla la medida del estimador del spline polinomial, el cual puede variar desde el modelo m\'as complicado e inestable hasta el modelo m\'as simple. En otras palabras, la ecuaci\'on (\ref{min}) representa un balance entre la fidelidad o ajuste de los datos, representado mediante la suma de los residuos al cuadrado y la suavidad de la curva resultante, la cual se representa por la integral del cuadrado de la m-\'emisa derivada.


\section{Citas}

%Para citar un libro o un art\'{\i}culo se hace as\'{\i} : \cite{ADRS} \cite{Ar}


\chapter{Metodolog\'ia.}

\section{Elaboraci\'on de la base de datos.}


\hspace{0.4cm} La fuente principal de informaci\'on para calcular la curva de rendimientos para los t\'itulos de la deuda p\'ublica nacional es el Banco Central de Venezuela (BCV), el cual diariamente publica las operaciones realizadas con estos instrumentos y los publica en el documento ``resumersec"\hspace{0.01cm}  (Ver Figura \ref{doc_bcv}). Es importante destacar, que en este documento se encuentran por d\'ia dos pesta\~nas, la $``0-22"$ y la $``0-23"$, en la primera pesta\~na se encuentran las operaciones interbancarias, por su parte la segunda pesta\~na muestra informaci\'on sobre las operaciones realizadas por entes privados, en este caso el precio pautado en la operaci\'on no est\'a disponible, raz\'on por la cual esta pesta\~na no se toma en consideraci\'on. La informaci\'on disponible en la pesta\~na $``0-22"$ es la siguiente,

\begin{itemize}
  \item C\'odigo del instrumento: c\'odigo \'unico que se asocia a cada instrumento.
  \item Fecha de vencimiento: fecha de maduraci\'on de cada instrumento.
  \item Plazo: cantidad de d\'ias que faltan para que el instrumento venza.
  \item Cantidad de Operaciones: n\'umero de operaciones realizadas con cada instrumento.
  \item Monto en Bol\'ivares: monto total involucrado en la operaci\'on.
  \item Precio m\'inimo: precio m\'as bajo pautado en la operaci\'on.
  \item Precio m\'aximo: precio m\'as alto pautado en la operaci\'on.
  \item Precio promedio: precio promedio pautado en la la operaci\'on. Cabe destacar que si existe una s\'ola operaci\'on, los precios m\'inimo, m\'aximo y promedio ser\'an iguales.
  \item Cup\'on: tasa de inter\'es pagadera por cada instrumento.
\end{itemize}


\vspace{0.5cm}

\begin{figure}[h]
  \scalebox{0.50}{\includegraphics{images/Imagen022.png}}
%\includegraphics[width=0.7\textwidth]{Imagen022.png}
\caption{Pesta\~na ``0-22".}
\label{doc_bcv}
\end{figure}

\vspace{0.5cm}

\hspace{0.4cm} Es importante recordar que dentro de los t\'itulos de la Deuda P\'ublica Nacional se encuentran los t\'itulos de inter\'es fijo (TIF) y los t\'itulos de tasa variable (VEBONO), los primeros se caracterizan por poseer una tasa de cup\'on que no varia, por su parte los VEBONO poseen una tasa de inter\'es variable.

\vspace{0.5cm}

\hspace{0.4cm}Esta informaci\'on tambi\'en es suministrada por el BCV, en su documento de las ``Caracter\'isticas de la Deuda P\'ublica Nacional" (Ver Figura \ref{doc_carac}), por lo cual el mismo se debe revisar con cierta frecuencia, con el fin de actualizar la tasa de cup\'on de los VEBONO. En este documento se muestra informaci\'on que caracteriza a cada instrumento, el mismo posee varias pesta\~nas, en este trabajo s\'olo se considerar\'a la pesta\~na $``DPN"$ en donde se encuentra informaci\'on sobre los instrumentos emitidos en moneda nacional. La informaci\'on disponible en este documento se muestra a continuaci\'on,

\begin{itemize}
  \item N\'umero-Emisi\'on-Decreto: informaci\'on sobre emisi\'on de cada instrumento.
  \item C\'odigo: c\'odigo \'unico que se asocia a cada instrumento.
  \item Fecha de emisi\'on: fecha cuando se emiti\'o cada instrumento.
  \item Fecha de vencimiento: fecha de maduraci\'on de cada instrumento.
  \item Monto en Circulaci\'on: monto total de cada instrumento en circulaci\'on.
  \item Porcentaje de referecia: indica si el instrumento es de tasa fija o tasa variable.
  \item Fecha de inicio: indica cada cuanto tiempo el instrumento paga cup\'on.
  \item Per\'iodo vigente: indica el per\'iodo (fecha inicio y fecha fin) cuando cada instrumento paga cup\'on.
  \item Tasa: cup\'on asociado a cada instrumento.
\end{itemize}



\begin{figure}[h]
  \scalebox{0.50}{\includegraphics{images/Imagencarac.png}}
\caption{Caracter\'isticas.}
\label{doc_carac}
\end{figure}


\hspace{0.4cm} A partir de la pesta\~na ``0-22"\hspace{0.01cm} y del documento de las caracter\'isticas, se cre\'o la base de datos con la cual se va a trabajar, la misma contiene no s\'olo la informaci\'on suministrada por la pesta\~na ``0-22", sino alguna informaci\'on adicional tomada del documento de las caracter\'isticas. En dicha base de datos se contar\'a con la siguiente informaci\'on,

\vspace{0.5cm}

\begin{itemize}
  \item Tipo Instrumento: Indica el tipo de instrumento.
  \item Nombre: Proporciona el nombre corto del t\'itulo, usualmente este nombre se conforma por el tipo de t\'itulo m\'as su mes y a\~no de vencimiento, por ejemplo, el t\'itulo TIF032028, representa al t\'itulo TIF con vencimiento en marzo del 2028.
  \item Fecha de operaci\'on: Indica la fecha en que se efectu\'o dicha operaci\'on.
  \item Fuente: Indica la fuente de donde se tom\'o la informaci\'on, esta se puede tomar de dos fuentes, la primera mediante la pesta\~na 0-22 (mercado secundario) y  la otra mediante el documento de las subastas (mercado primario, informaci\'on suministrada por el BCV).
  \item Sicet: Proporciona el c\'odigo asociado a cada t\'itulo.
  \item Fecha de vencimiento: Indica la fecha de maduraci\'on (vencimiento) del instrumento.
  \item Plazo: Indica la cantidad de d\'ias que falta para que el instrumento se venza.
  \item Cantidad de operaciones: Proporciona la cantidad de operaciones efectuadas con un insrumento en espec\'ifico.
  \item Monto: Indica el monto en Bol\'ivares, por el cual se efectu\'o la operaci\'on u operaciones.
  \item Precio m\'inimo: Indica el precio m\'inimo, por el cual se trans\'o la operaci\'on.
  \item Precio m\'aximo: Indica el precio m\'aximo, por el cual se trans\'o la operaci\'on.
  \item Precio promedio: Indica el precio promedio, por el cual se trans\'o la operaci\'on, cabe destacar que en dado caso de existir una sola operaci\'on el valor del precio m\'inimo, m\'aximo y promedio van a coincidir.
  \item Cup\'on: Proporciona la tasa de cup\'on asociado a cada instrumento.
  \item Frecuencia: Indica con que frecuencia el instrumento paga cup\'on, para los TIF y VEBONO, esta es 4, pues los mismos pagan cu\'pon trimestralmente, as\'i se obtiene este valor pues existen 4 trimestres en el a\~no.

\end{itemize}

\vspace{0.5cm}

\hspace{0.4cm} Una vez obtenida la base de datos esta seg\'un sea el caso puede ser depurada mediante ciertos criterios, el primero es que aquellas operaciones con un monto menor a los 10 milllones no se consideran. El segundo es considerar la operaci\'on mas reciente, es decir, si en la base de datos se tiene que para un mismo instrumento existen diferentes operaciones en diferentes d\'ias, s\'olo se considerar\'a la operaci\'on m\'as reciente.

%El segundo es el tipo de fuente, siempre prevalecer\'a la fuente subasta.

\hspace{0.4cm} Para efectuar la depuraci\'on, a la base de datos anterior se le a\~nadir\'an dos columnas nuevas una que indica el rendimiento al vencimiento de cada instrumento y la otra que indica la decisi\'on que se tom\'o en base a los criterios descritos anteriormente (Ver Figura \ref{base_datos}). Esta \'ultima ser\'a una variable dicot\'omica, es decir solo con dos valores (0 \'o 1), en donde ``0" me indica que no selecciono el t\'itulo y ``1" me indica que si lo tomo en cuenta para el estudio a realizar.

\begin{figure}[h]
  \scalebox{0.50}{\includegraphics{images/Imagenbase.png}}
\caption{Base de datos.}
\label{base_datos}
\end{figure}


\hspace{0.4cm} Una vez calculados los precios estimados asociados a cada instrumento, se proceder\'a a comparar los mismo con aquellos obtenidos por una metodolog\'ia distinta. La metodolog\'ia con la cual se va a comparar es la de Svensson, la cual es una metodolog\'ia param\'etrica.



\hspace{0.4cm} Los instrumentos a considerar ser\'an aquellos pertencientes al portafolio de inversiones de una instituci\'on financiera, de tal manera que para un d\'ia especifico sea posible conocer cuanta es la ganancia o p\'erdida que generan estos instrumentos. y por ende saber si es viable la venta o compra de determinado instrumento.


\hspace{0.4cm} A partir de la data obtenida (Ver Figura \ref{base_datos}), se proceder\'a a a\~nadir unas columnas nuevas con el fin de clasificar las observaciones para los distintos instrumentos en diferentes per\'iodos de vencimiento. Los per\'iodos de vencimiento son,

\begin{itemize}
\item Corto plazo: se refiere al vencimiento m\'as cercano, los instrumentos que se encuentran aqu\'i son aquellos que poseen un vencimiento menor a un a\~no.
\item Mediano plazo: en esta clasificaci\'on se encuentran los instrumentos cuyo vencimiento este entre uno y diez a\~nos.
\item Largo plazo: hace referencia a aquellos instrumentos que tengan un vencimiento mayor a diez a\~nos.
\end{itemize}

\hspace{0.4cm} Luego de separar la data por tipo de instrumento, la nueva data con la que se trabajar\'a es la siguiente,

\begin{figure}[h]
  \scalebox{0.50}{\includegraphics{images/data_nueva.png}}
\caption{Base de datos TIF.}
\label{base_datos_tif}
\end{figure}

\hspace{0.4cm} Con el fin de contar con la data m\'as reciente a partir de la fecha de valoraci\'on, se cre\'o la funci\'on ``extrae" \hspace{0.01cm} la cual selecciona de la data de la Figura (\ref{base_datos_tif}) una determinada cantidad de observaciones, la cual es especificada por el usuario, esta funci\'on cuenta con los siguientes argumentos,

\begin{itemize}
 \item fv: indica la fecha de valoraci\'on para la cual se est\'a realizando el estudio.
 \item dias: indica la cantidad de d\'ias que el usuario desea, a partir de este valor se va a obtener la data con la que se va a trabajar.
 \item data: hace referencia a la data completa para cada tipo de instrumento, a partir de la misma se procedera a extraer parte de ella a partir del n\'umero de dias seleccionado.
\end{itemize}

\hspace{0.4cm} Luego de selecionar la data, la misma se procede a depurar, es decir, se van a eliminar las observaciones duplicadas considerando s\'olo aquellas que sean m\'as recientes. 


\hspace{0.4cm}As\'i a partir de esta data s\'olo se consideraran las columnas plazo y rendimiento con el fin de tener una nube de puntos a partir de la cual se haga el ajuste de la funci\'on spline, y as\'i obtener la curva de rendimientos.

\newpage

\hspace{0.4cm} La data obtenida a partir de la depuraci\'on anterior es,

\begin{figure}[h]
    {\includegraphics{images/cand.png}}
\caption{Data depurada TIF.}
\label{fig91}
\end{figure}



\hspace{0.4cm} Una vez obtenida la data para los Tif y Vebono se utiliz\'o la funci\'on ``smooth.spline" \hspace{0.01cm} del programa estad\'istico R, para ajustar un spline c\'ubico a la data ingresada. Los argumentos requeridos por esta funci\'on son los siguientes,

\begin{itemize}
  \item X: representa el vector de la variable predictiva.
  \item Y: representa el vector de la variable repuesta.
  \item cv: (TRUE/FALSE) variable del tipo l\'ogico que representa si se va a utilizar la validaci\'on cruzada generalizada al momento de calcular el par\'ametro de suavizamiento.
  \item Spar: representa el par\'ametro de suavizamiento, t\'ipicamente (aunque no necesariamente) ubicado entre 0 y 1. Es el coeficiente lambda que acompa\~na a la integral del cuadrado de la segunda derivada de la funci\'on f.
\end{itemize}

\hspace{0.4cm} De esta manera el siguiente comando ajusta un spline cubico a la data ingresada,


spline1=smooth.spline(X=datT1\$Plazo,Y=datT1\$Rendimiento,cv=TRUE, spar=1.35)


\noindent y lo guarda en la variable ``spline1".

\hspace{0.4cm} Es importante se\~nalar lo crucial de la escogencia del par\'ametro ``spar", pues de \'el depende que tan suave sea la curva, la Figura \ref{comp_spar} muestra como var\'ia la curva cuando se cambia  el valor del ``spar", para esta comparaci\'on se us\'o tres valores, el primero fue 0.51 con el cual se obtiene una curva con ciertos picos la cual no es suave en lo absoluto. 

\newpage

\hspace{0.4cm} Usando el valor de 0.71 se obtiene la curva roja la cual presenta una mayor suavidad. Mientras que usando el valor de 0.81 se obtiene un mejor resultado aunque similar al anterior. De esta manera, se puede observar la importancia de la elecci\'on correcta de este par\'ametro, mientras este valor se aproxime a 1 se obtendr\'a una curva con mayor suvidad.

\begin{figure}[h]
  {\includegraphics{images/curvavbv2.jpg}}
\caption{Curva de rendimiento Vebono para diferentes valores de suavizado.}
\label{comp_spar}
\end{figure}

\hspace{0.4cm} Cabe destacar que para cada versi\'on el par\'ametro usado en la variable ``spar" \hspace{0.2cm} cambi\'o. Esto debido a la diferente cantidad de puntos que tiene cada versi\'on. As\'i el valor del par\'ametro ``spar" \hspace{0.2cm} para los TIF se ubic\'o en el siguiente intervalo [0.4,0.6], por su parte para los VEBONOS la elecci\'on de dicho par\'ametro esta en [0.3,0.5]. Los mismos se obtuvieron mediante ensayo y error. Para los valores ubicados dentro de los intervalos mencionados siempre se obtuvo una curva suave.

\hspace{0.4cm} Una vez que se obtiene la curva estimada y es guardada en una variable (en este caso, la variable es spline1), se procede a aplicar el comando ``predict", para estimar el rendimiento de alg\'un plazo que se ingrese.

\hspace{0.4cm} As\'i con el fin de calcular el precio estimado de cada t\'itulo, se cre\'o la funci\'on ``precio"\hspace{0.01cm} mediante R, para determinar de forma autom\'atica dichos valores. Los imputs de dicha funci\'on son los siguientes,


\begin{itemize}
  \item Tit: representa el nombre de cada t\'itulo, al cual se le quiere estimar su precio, el mismo debe ser un car\'acter, ej: TIF102017 \'o VEBONO112017.
  \item Spline1: representa la variable donde se guardo la curva ajustada mediante el spline.
  \item Fv: indica la fecha de valoraci\'on, para la cual se desea conocer el precio estimado.
\end{itemize}


\hspace{0.4cm} Una vez ingresado los imputs, la funci\'on internamente busca el nombre del t\'itulo en el documento de las caracter\'isticas m\'as reciente, y extrae del mismo la fecha de pago del pr\'oximo cup\'on y su fecha de vencimiento, con el fin de crear un vector de flujos.


\hspace{0.4cm} Por ejemplo, si se quiere conocer el precio estimado del t\'itulo ``TIF032022" \hspace{0.01cm} al ``01/03/2018", la funci\'on busca su fecha de vencimiento $(03/03/2022)$ y la fecha de pago del pr\'oximo cup\'on la cual es en este caso $08/03/2018$. Luego con dichos valores calcula la Tabla \ref{tabla1}, que representa los cupones que le quedan por pagar al t\'itulo,

\renewcommand{\tablename}{Tabla}
\begin{table}[H]
\centering
%\begin{center}
{\begin{tabular}[t]{|l |c |c |c |c |c |r|}
\hline
Fecha & Plazo t\'itulo & Plazo a\~nos & Rend estimado & Exp & Cup\'on & Producto \\
\hline
08/03/2018 & 7  & 0,0191780 & 0,45\% & 0,9999131 & 4& 3,999652\\
\hline
07/06/2018 & 98 & 0,2684931 & 1,05\% & 0,9971804 & 4& 3,988721\\
\hline
06/09/2018 & 189 & 0,5178082 & 1,64\% & 0,9915025 & 4& 3,966010\\
\hline
06/12/2018 & 280 & 0,7671232 & 2,23\% & 0,9829646 & 4& 3,931859\\
\hline
07/03/2019 & 317 & 1,0164383 & 2,82\% & 0,9717013 & 4& 3,886805\\
\hline
06/06/2019 & 462 & 1,2657534 & 3,39\% & 0,9578928 & 4& 3,831571\\
\hline
05/09/2019 & 553 & 1,5150684 & 3,96\% & 0,9417596 & 4& 3,767038\\
\hline
05/12/2019 & 644 & 1,7643835 & 4,50\% & 0,9235567 & 4& 3,694227\\
\hline
05/03/2020 & 735 & 2,0136986 & 5,03\% & 0,9035668 & 4& 3,614267\\
\hline
04/06/2020 & 826 & 2,2630137 & 5,54\% & 0,8820934 & 4& 3,528373\\
\hline
03/09/2020 & 917 & 2,5123287 & 6,02\% & 0,8594532 & 4& 3,437813\\
\hline
03/12/2020 & 1008 & 2,7616438 & 6,48\% & 0,8359698 & 4& 3,343879\\
\hline
04/03/2021 & 1099 & 3,0109589 & 6,91\% & 0,8119665 & 4& 3,247866\\
\hline
03/06/2021 & 1190 & 3,2602739 & 7,31\% & 0,7877226 & 4& 3,150890\\
\hline
02/09/2021 & 1281 & 3,5095890 & 7,69\% & 0,7634473 & 4& 3,053789\\
\hline
02/12/2021 & 1372 & 3,7589041 & 8,03\% & 0,7393192 & 4& 2,957277\\
\hline
03/03/2022 & 1463 & 4,0082191 & 8,35\% & 0,7154912 & 104 & 74,411084\\
\hline
% Precio &  &  &  & & & 112,688809\\
\multicolumn{6}{|c|}{Precio} & 131,8111 \\
\hline
\end{tabular}
}
%\caption{Tabla}
%\end{center}
\caption{C\'alculos funci\'on precio.}
\label{tabla1}
\end{table}

\hspace{0.4cm}As\'i la primera columna (Fecha) se obtiene de sumarle a la fecha de pago del pr\'oximo cup\'on ($08/03/2018$) 91 d\'ias, que representa el tiempo cada cuando el t\'itulo paga cup\'on, esto se realiza  hasta llegar a la fecha de vencimiento.

\hspace{0.4cm} Luego la columna ``Plazo t\'itulo", se obtiene realizando la diferencia entre la columna 1 y la fecha de valoraci\'on (01/03/2018). Luego la columna 3 se obtiene dividiendo el valor de la columna 2 entre 365, para pasar dicho valor a a\~nos. Despu\'es eval\'uo los valores de la columna 2 en el spline obtenido, para as\'i obtener los rendimientos estimados (columna 4). Posteriormente en la columna 5 (EXP) calculo la exponencial del producto de menos uno con el plazo en a\~nos (columna 3) y con el rendimiento estimado (columna 4).


\hspace{0.4cm} La columna 6 (Cup\'on) la calculo dividiendo el valor del cup\'on del t\'itulo entre 4, ya que cada cup\'on se paga cada tres meses, a diferencia del \'ultimo al cual se le debe sumar el valor de 100. Finalmente en la \'ultima columna (Producto) calculo el producto del valor de la columna EXP con la columna Cup\'on, para luego realizar la sumatoria de todas sus filas y as\'i obtener el precio estimado ( 131,8111 en este caso).


\hspace{0.4cm}El mismo procedimiento se repite para cada t\'itulo ya sea Tif o Vebono. Es importante se\~nalar que los t\'itulos considerados fueron aquellos que pertenec\'ian al portafolio de inversiones del banco en un tiempo determinado.

\section{Estimaci\'on de par\'ametros y curva de rendimiento.}

\hspace{0.4cm} Una vez construida la base de datos, se proceder\'a a utilizar los splines de suavizado para obtener los par\'ametros necesarios para la curva de rendimientos. Recordemos que esta curva relaciona el plazo del instrumento con su rendimiento.


\hspace{0.4cm} Es importante se\~nalar que se estimar\'a una curva por cada tipo de instrumento, as\'i se obtendr\'a un curva para los TIF y una curva para los VEBONO. Por tal raz\'on a partir de la base de datos, se separar\'a los TIF de los VEBONOS, y se considerar\'an s\'olo las columnas Plazo y Rendimiento para estimar dicha curva. Seg\'un sea el caso, s\'olo considerar\'an aquellas observaciones que tengan decisi\'on 1.


\hspace{0.4cm} Aunado a cada tipo de instrumento (TIF \'o VEBONO), se considerar\'a un instrumento de otro tipo este es la letra del tesoro, este tipo de instrumento representar\'a el punto inicial la curva, cabe destacar que la letra a considerar debe ser aquella cuya fecha de operaci\'on sea la m\'as reciente con respecto a la fecha de valoraci\'on (d\'ia en que se quiere conocer los rendimientos estimados).


\hspace{0.4cm} A partir de la curva de rendimientos obtenida (Ver Figura \ref{c_rend}) es posible calcular un rendimiento estimado para alg\'un tipo de instrumento a partir de su plazo, que no es m\'as que la cantidad de d\'ias que faltan por transcurrir hasta su vencimiento. Este valor es de suma importancia ya que a partir del mismo es posible calcular el precio estimado asociado a cada instrumento en un d\'ia espec\'ifico. Con lo cual es posible saber a partir de la historia (base de datos), el precio estimado de alg\'un instrumento que le interese a cierta instituci\'on y por ende saber si ese t\'itulo es rentable o no, es decir, si vale la pena invertir en el mismo o no.

\begin{figure}[h]
  \scalebox{0.40}{\includegraphics{images/curvarend.jpeg}}
\caption{Curva de Rendimiento.}
\label{c_rend}
\end{figure}

\hspace{0.4cm} Como se dijo anteriormente, los resultados de los precios obtenidos mediante el uso de la metodolog\'ia de splines de suavizado ser\'an comparados con los precios obtenidos a trav\'es de la metodolog\'ia de Svensson. En dicha metodolog\'ia existe un proceso de optimizaci\'on el cual permite encontrar los par\'ametros id\'oneos, de tal manera que la diferencia entre los precios promedio de cada instrumento y su precio te\'orico sea lo m\'as peque\~na posible. El proceso de esta optimizaci\'on se muestra a continuaci\'on.


%\section{Elecci\'on \'optima del par\'ametro de suavizamiento.}

\newpage

\section{Proceso de Optimizaci\'on de Svensson.}


\hspace{0.4cm} Para aplicar este proceso es necesario tener una funci\'on objetivo, sobre la cual se realizar\'a el proceso de optimizaci\'on, ya sea para maximizar \'o minimizar dicha funci\'on. Dependiendo de la forma de dicha funci\'on el proceso de optimizaci\'on ser\'a lineal o no lineal. En nuestro caso particular se llevar\'a a cabo un proceso de optimizaci\'on no lineal donde se buscar\'a minimizar la funci\'on objetivo.


\hspace{0.4cm} En el c\'alculo de nuestra funci\'on objetivo inteviene el concepto de la duraci\'on de un bono \'o t\'itulo, la cu\'al es una medida del vencimiento medio ponderado de todos los flujos que paga un bono. La misma viene dada mediante la siguiente expresi\'on, \\

\begin{center}

$\displaystyle{Duracion = \frac{1+r}{r} - \frac{n(c-r)+(1+r)}{c(1+r)^{n}-(c-r)}}$

\end{center}


\noindent donde

\begin{itemize}
  \item r es el rendimiento al vencimiento del bono durante el per\'iodo considerado.
  \item n es el n\'umero de per\'iodos que restan hasta la fecha de vencimiento del bono.
  \item c es el cup\'on del bono.
\end{itemize}


\hspace{0.4cm} As\'i nuestra funci\'on objetivo viene dada mediante la siguiente expresi\'on,

\vspace{0.2cm}

\begin{equation}\label{ecua2}
  f(x) = \sum_{i=1}^{n} (w_{i}\epsilon(x)_{i} )^2
\end{equation}


\noindent donde $w_{i}$ representan las ponderaciones, y se calculan mediante la siguiente expresi\'on,

\vspace{0.2cm}


\begin{center}

$\displaystyle{w_{i} = \frac{\frac{1}{D_{i}}}{\sum_{j=1}^{N}\frac{1}{D_{j}}}}$

\end{center}


\vspace{0.2cm}


\noindent por su parte, $\epsilon_{i}(x)= \hat{Pr}_{i}(x)-Pr_{i}$, donde $Pr_{i}$ representan los precios promedios de los t\'itulos a considerar, de entrada este es un par\'ametro \'o valor con el que se cuenta. Por otra parte $\hat{Pr}_{i}(x)$ representa los precios estimados donde $x$ es el par\'ametro que va a variar y es el valor que se quiere optimizar.


\hspace{0.4cm} Mediante la funci\'on objetivo descrita anteriormente se busca minimizar la diferencia que existe entre los precios promedios y los precios estimados, calculando un valor \'optimo del par\'ametro $x$ mediante el proceso de optimizaci\'on no lineal.



\hspace{0.4cm}El proceso de optimizaci\'on se realiz\'o mediante el software estad\'istico R, mediante el paquete ``nloptr". En este paquete, se encuentra el comando ``aulag" el cual minimiza un funci\'on objetivo y devuelve entre otros valores el par\'ametro m\'as \'optimo, que hace que la funci\'on sea m\'inima. Un ejemplo del uso de este comando se presenta acontinuaci\'on,

\begin{center}
  $ala2=auglag(1.22, fn=mifuncion, hin=res)$
\end{center}


\noindent donde el primer argumento debe ser el valor inicial del par\'ametro a optimizar, el segundo argumento ``fn" se refiere a la funci\'on que se desea optimizar, finalmente en el tercer par\'ametro ``hin" se indican las restricciones sobre el par\'ametro a optimizar, en este caso la restricci\'on establecida es que el par\'ametro sea mayor a cero.


\hspace{0.4cm} Recordemos que la tasa cero cup\'on que se obtiene mediante la metodolog\'ia de Svensson est\'a dada por la siguiente expresi\'on,\\


$\displaystyle{s(m) = \beta_{0}+ \beta_{1}\frac{\left(1-e^\frac{-m}{\tau_{1}}\right)}{m/\tau_{1}} + \beta_{2} \left(\frac{\left(1-e^\frac{-m}{\tau_{1}}\right)}{m/\tau_{1}} -  e^\frac{-m}{\tau_{1}}\right) + \beta_{3} \left(\frac{\left(1-e^\frac{-m}{\tau_{2}}\right)}{m/\tau_{2}} -  e^\frac{-m}{\tau_{2}}\right)}$\\

\noindent esta expresi\'on est\'a sujeta a las siguientes restricciones,

\begin{itemize}
  \item $\beta_{0} > 0$
  \item $\beta_{0}+\beta_{1} > 0$
  \item $\tau_{1} > 0$
  \item $\tau_{2} > 0$
\end{itemize}

\noindent cada par\'ametro controla una secci\'on de la curva. La f\'ormula anterior es de suma importancia ya que ella interviene en el c\'alculo del precio te\'orico de cada instrumento. El proceso de optimizaci\'on act\'ua directamente sobre esta f\'ormula, ya que el mismo se centra en variar los par\'ametros de tal manera que la funci\'on objetivo sea minimizada.

\hspace{0.4cm} Como se observ\'o en las secciones anteriores el par\'ametro de suavizamiento fu\'e elegido mediante el m\'etodo de ensayo y error el cual no es para nada \'optimo pues a priori este m\'etodo no nos garantiza que el valor seleccionado sea el mejor, ya que se contar\'ian con una gran cantidad de posibles valores a seleccionar, con el fin  de encontrar dicho par\'ametro el procedimiento anteriormente explicado puede ser implementado. 

\hspace{0.4cm} Sin embargo, al realizar este proceso, se obtienen curvas que no son para nada suaves y en ocasiones no poseen ningunas de las formas usuales de la curva de rendimientos. Esto es debido a que en este caso este proceso, var\'ia el par\'ametro de suavizamiento de tal manera que la diferencia entre el precio promedio y el precio te\'orico sea lo mas peque\~na posible y en este proceso no existe un par\'ametro que controle la forma de la curva obtenida. Por lo tanto, su aplicaci\'on presenta algunos inconvenientes.  







\chapter{Resultados y conclusiones.}

\section{Comparaci\'on TIF.}

\hspace{0.4cm} La comparaci\'on de los precios te\'oricos obtenidos mediante el uso de la metodolog\'ia de los splines c\'ubicos de suavizado con la metodolog\'ia Svensson para los Tif, se presenta a continuaci\'on en la Tabla \ref{tabla2}.

\renewcommand{\tablename}{Tabla}
\begin{table}[H]
\centering
%\begin{center}
\scalebox{0.85}{\begin{tabular}[t]{|c |c |c |c |c |c |r|}
\hline
T\'itulo & Precio promedio & Precio splines & Precio Svensson & Precio Svensson optimizado  \\
\hline
TIF082018 & 101,00 & 107,27 & 107,35 & 100,92  \\
\hline
TIF042019 & 112,00 & 116,38 & 119,48 & 113,41  \\
\hline
TIF082019 &  110,00& 109,42 & 112,81 & 108,19 \\
\hline
TIF112020 & 130,02 & 127,62 & 130,15 & 127,28  \\
\hline
TIF022021 & 129,01 & 128,05 & 130,37 & 127,56  \\
\hline
TIF042023 & 128,10 & 132,21 & 134,03 & 130,27  \\
\hline
TIF012024 & 120,00 & 134,42 & 136,25 & 132,02 \\
\hline
TIF062025 & 124,00 & 130,44 & 131,61 & 126,85  \\
\hline
TIF012026 & 122,00 & 130,31 & 131,05 & 126,04  \\
\hline
TIF112027 & 126,52 & 132,04 & 130,75 & 125,08 \\
\hline
TIF032028 & 128,52 & 135,83 & 134,02 & 128,15  \\
\hline
TIF052028 & 128,19 & 137,56 & 131,76 & 125,88  \\
\hline
TIF032029 & 132,03 & 140,30 & 137,37 & 131,18  \\
\hline
TIF022030 & 128,52 & 142,86 & 138,91 & 132,42  \\
\hline
TIF022031 & 130,10 & 135,95 & 131,54 & 125,01  \\
\hline
TIF032031 & 128,53 & 138,30 & 133,87 & 127,30  \\
\hline
TIF022032 & 127,00 & 135,13 & 127,48 & 120,92\\
\hline
TIF032032 & 128,52 & 139,45 & 135,46 & 128,60  \\
\hline
TIF032033 & 127,01& 135,37 & 130,67 & 123,91\\
\hline
TIF052034 & 127,12 & 128,51 & 124,08 & 117,35\\
\hline
Error & NA & 4,6224 & 5,2694 & 0,3610\\
\hline
\end{tabular}}
%\end{center}
\caption{Comparaci\'on precios TIF.}
\label{tabla2}
\end{table}


\hspace{0.4cm} Para la creaci\'on de la tabla anterior, se consideraron los siguientes par\'ametros,

\begin{itemize}
  \item Fecha de valoraci\'on: $08/03/2018$.
  \item D\'ias hacia atras: 40.
  \item Par\'ametro de suavizamiento: 0.2
\end{itemize}


\hspace{0.4cm} En la tabla anterior se observan los precios promedios de cada instrumento para el a\~no en curso, los precios te\'oricos obtenidos mediante la metodolog\'ia de splines y los precios te\'oricos obtenidos mediante la metodolog\'ia de Svensson, en este caso se tienen dos resultados uno es usando unos par\'ametros por defecto (columna precio Svensson) y el otro es usando el proceso de optimizaci\'on para esta metodolog\'ia. En la \'ultima fila se observa el valor de la suma del error cuadr\'atico (Error), el cual se obtiene luego de realizar la suma de los cuadrados de las diferencias que existen entre el precio te\'orico de cada instrumento y su precio promedio. Mientras mas peque\~no sea este valor m\'as se asemejar\'an los precios te\'oricos a los precio promedio.

\hspace{0.4cm} Si se comparan los precios te\'oricos obtenidos mediante la metodolog\'ia de splines y aquellos obtenidos mediante la metodolog\'ia de Svensson (sin optimizar), se puede afirmar que mediante la primera metodolog\'ia se obtiene una mejor aproximaci\'on de dichos precios con respecto a los precios promedio de cada instrumentos esto debido al error que para el caso de los splines es de $4.6224$, mientras que para el caso de Svensson es de $5.2694$.

\hspace{0.4cm} Por su parte, si se observa el error obtenido para la metodolog\'ia de Svensson optimizada ($0.3610$), este valor es mucho m\'as peque\~no que las dos metodolog\'ias anteriores, esto se debe al proceso de optimizaci\'on ya que en el mismo se busca que esta diferencia sea m\'inima.

\hspace{0.4cm} La Figura \ref{curva_spline_tif}, muestra la curva de rendimiento para los TIF obtenida mediante el ajuste del spline de suavizado, por su parte la nube de puntos fue obtenida a partir de la base de datos de estos instrumentos para la cantidad de d\'ias seleccionado, en este caso 40. Estos puntos representan los rendimientos obtenidos a partir de las operaciones realizadas con estos instrumentos en el horizonte temporal considerado. El eje x de la gr\'afica muestra la maduraci\'on en d\'ias, por su parte el eje y muestra el rendimiento estimado.

\newpage

\section{Comparaci\'on VEBONO.}


\hspace{0.4cm}La siguiente tabla muestra los resultados de los precios te\'oricos obtenidos para los Vebonos mediante las diferentes metodolog\'ias consideradas. Para la creaci\'on de la misma, se consideraron los siguientes par\'ametros,

\begin{itemize}
  \item Fecha de valoraci\'on: $08/03/2018$.
  \item D\'ias hacia atras: 40.
  \item Par\'ametro de suavizamiento: 0.4
\end{itemize}

%\begin{center}
\begin{table}[H]
\centering
\scalebox{0.90}{\begin{tabular}[t]{|c |c |c |c |c |c |r|}
\hline
T\'itulo & Precio promedio & Precio splines & Precio Svensson & Precio Svensson optimizado  \\
\hline
VEBONO072018 & 100,40 & 106,17 & 108,14 & 100,41  \\
\hline
VEBONO022019 & 106,00 & 107,75 & 111,18 & 104,98  \\
\hline
VEBONO032019 &  110,00& 113,55 & 117,10 & 111,38 \\
\hline
VEBONO012020 & 121,00 & 118,42 & 121,05 & 118,92  \\
\hline
VEBONO062020 & 127,83 & 120,97 & 122,60 & 122,23  \\
\hline
VEBONO012021 & 130,32 & 121,96 & 122,27 & 123,36  \\
\hline
VEBONO052021 & 127,00 & 122,01 & 121,72 & 123,28 \\
\hline
VEBONO122021 & 129,45 & 126,18 & 124,86 & 126,88  \\
\hline
VEBONO022022 & 129,00 & 123,15 & 121,55 & 123,62 \\
\hline
VEBONO012023 & 129,96 & 126,79 & 123,69 & 125,78 \\
\hline
VEBONO022024 & 128,00 & 128,27 & 123,88 & 125,77  \\
\hline
VEBONO022025 & 128,50 & 134,25 & 125,35 & 126,96  \\
\hline
VEBONO042028 & 129,68 & 132,08 & 127,80 & 128,62  \\
\hline
VEBONO102028 & 130,01 & 131,67 & 127,68 & 128,40  \\
\hline
VEBONO052029 & 125,03 & 131,16 & 127,26 & 127,91  \\
\hline
VEBONO102029 & 125,75 & 132,32 & 128,24 & 128,79 \\
\hline
VEBONO072030 & 130,50 & 132,44 & 127,72 & 128,16\\
\hline
VEBONO062032 & 128,53 & 128,07 & 121,73 & 121,89  \\
\hline
VEBONO072033 & 130,00& 127,31 & 120,40 & 120,45\\
\hline
VEBONO022034 & 128,02 & 126,53 & 119,49 & 119,51\\
\hline
Error & NA & 3,5817 & 6,7379 & 0,3608\\
\hline
\end{tabular}}
%\end{center}
\caption{Comparaci\'on precios VEBONO.}
\label{tabla3}
\end{table}

\newpage

\hspace{0.4cm} Al igual que en la tabla comparativa para los TIF, la Tabla \ref{tabla3} muestra los precios te\'oricos obtenidos por tres metodolog\'ias, la primera la de splines de suavizado, la segunda la de Svensson sin optimizar y finalmente la tercera, la de Svensson optimizado. Estas metodolog\'ias, son comparadas con los precios promedio para cada instrumento.

\hspace{0.4cm} Si se compara los precios te\'oricos obtenidos mediante la metodolog\'ia de splines de suavizados con los precios obtenidos mediante la metodolog\'ia de Svensson sin optimizar, podemos decir que los primeros muestran un mejor comportamiento al ser comparados con los precios promedio de cada t\'itulo valor. Esto debido a que el error obtenido es menor.

\hspace{0.4cm}Por otra parte, si estos precios son comparados con los obtenidos mediante la metodolog\'ia de Svensson optimizada, se observa que estos \'ultimos poseen un excelente comportamiento, ya que su error es bastante peque\~no. En la Figura \ref{curva_spline_veb} se observa la curva de rendimientos obtenida para la fecha de valoraci\'on y cantidad de d\'ias seleccionado. En este caso la nube de puntos, al igual que para los TIF surgue de la base de datos de las operaciones m\'as recientes transadas con este tipo de instrumentos.

\subsection{Curvas de Rendimiento.} \hspace{5cm}


\begin{figure}[h]
  \scalebox{0.80}{\includegraphics{images/c_tif.jpg}}
%\includegraphics[width=0.7\textwidth]{Imagen022.png}
\caption{Curva spline TIF.}
\label{curva_spline_tif}
\end{figure}


\begin{figure}[h]
  \scalebox{0.80}{\includegraphics{images/c_veb.jpg}}
%\includegraphics[width=0.7\textwidth]{Imagen022.png}
\caption{Curva spline VEBONO.}
\label{curva_spline_veb}
\end{figure}

\newpage

\section{Comparativo curvas de rendimiento.}

\hspace{0.4cm} En las Figuras \ref{curva_spline_comp_tif} y \ref{curva_spline_comp_veb} , se muestra una comparaci\'on entre las curvas de rendimiento obtenidas mediante la metodolog\'ias de splines de suavizado, Svensson sin opmitizar y Svensson optimizado.
Para el caso de la metodolog\'ia de los splines se us\'o un par\'ametro de suavizamiento de 0.2 para los TIF y 0.4 para los Vebono, recordemos que este factor controla la suavidad de la curva obtenida. Por su parte las curvas obtenidas para las metodolog\'ias de Svensson optimizada y sin optimizar surgen de los par\'ametros considerados al momento de calcular los precios te\'oricos de los instrumentos.

\hspace{0.4cm}Recordemos que para la metodolog\'ia Svensson sin optimizar se usaron unos par\'ametros por defecto, mientras que para la metodolog\'ia de Svensson optimizada se usaron aquellos par\'ametros de minimizaran la diferencia entre los precios te\'oricos y los precio promedio. La Tabla \ref{tabla4} muestra una comparaci\'on con estos valores para el caso de los TIF .

\begin{table}[H]
\centering
{\begin{tabular}[t]{|c |c |c |c |c |c |c |c |r|}
\hline
Metodolog\'ia / Par\'ametro / TIF & $\beta_{0}$ & $\beta_{1}$ & $\beta_{2}$ & $\beta_{3}$  &  $\tau_{1}$ & $\tau_{2}$ \\
\hline
Svensson sin optimizar & 0,1337 & -0,0100 & -0,3078 & -0,1340  & 0,5453 & 0,3506\\
\hline
Svensson Optimizado & 0,1430 & 0,2119 & -0,6438 & 0,3327 & 0,5744 & 0,0184 \\
\hline
\end{tabular}}
\caption{Par\'ametros Svensson TIF.}
\label{tabla4}
\end{table}


\newpage

\begin{figure}[h]
  \scalebox{0.65}{\includegraphics{images/Comparativo_tif.png}}
%\includegraphics[width=0.7\textwidth]{Imagen022.png}
\caption{Curva TIF spline vs Svensson.}
\label{curva_spline_comp_tif}
\end{figure}

\hspace{0.4cm} En la Figura \ref{curva_spline_comp_tif} se aprecia que la curva obtenida mediante la metodolog\'ia de los splines c\'ubicos de suavizado (curva roja) resulta ser la curva que menos altura alcanza en promedio siendo su valor m\'aximo $12 \%$, lo cual influye directamente en los precios obtenidos, esto debido a la relaci\'on inversa que existe entre el rendimiento y el precio de un instrumento. En este caso, al obtenerse rendimientos relativamente bajos se obtendr\'an precios relativamente elevados. 

\hspace{0.4cm} Por su parte si se observa la curva de rendimientos obtenida mediante la metodolog\'ia de Svensson sin optimizar (curva verde), la misma presenta alturas m\'as grandes que la curva anterior, alcanzando su m\'aximo en $12,5 \%$. Esto ocasiona, que para esta metodolog\'ia se obtengan precios un tanto m\'as bajos.

\hspace{0.4cm} Finalmente si se observa la curva de rendimientos obtenida mediante la metodolog\'ia de Svensson optimizada (curva azul), vemos que ella presenta las alturas m\'as elevadas, alcanzando su valor m\'aximo en $13 \%$ . De esta manera, los precios a obtener ser\'an m\'as bajos y en este caso los precios que m\'as se asemejan a los precios te\'oricos de los instrumentos TIF.




\newpage

\hspace{0.4cm} La tabla \ref{tabla5} muestra los par\'ametros utilizados para elaborar las curvas de rendimientos para el caso de la metodolog\'ia de Svensson sin optimizar y Svensson optimizado. Es importante destacar, que cada par\'ametro controla una secci\'on de la curva.

\begin{table}[H]
\centering
{\begin{tabular}[t]{|c |c |c |c |c |c |c |c |r|}
\hline
Metodolog\'ia / Par\'ametro / VEBONO & $\beta_{0}$ & $\beta_{1}$ & $\beta_{2}$ & $\beta_{3}$  &  $\tau_{1}$ & $\tau_{2}$ \\
\hline
Svensson sin optimizar & 0,1358 & 0,1000 & -0,5037 & -0,2887 & 0,1195 & 0,5017\\
\hline
Svensson Optimizado & 0,1421 & 0,4845 & -0,4343 & -0,3514 & 0,1985 & 0,7851 \\
\hline
\end{tabular}}
\caption{Par\'ametros Svensson VEBONO.}
\label{tabla5}
\end{table}

\vspace{0.5cm}

\hspace{0.4cm} La Figura \ref{curva_spline_comp_veb} muestra una comparaci\'on entre las curvas obtenidas por las tres metodolog\'ias consideradas. La curva roja muestra la curva de rendimientos obtenida mediante la metodolog\'ia de splines c\'ubicos de suavizado, la misma presenta ciertas ``jorobas", las cuales son ocasionadas por las observaciones obtenidas para el per\'iodo de estudio considerado. Esta curva presenta en promedio alturas m\'as bajas que las curvas obtenidas por las otras metodolog\'ias, raz\'on por la cual sus precios tender\'an a ser m\'as elevados.

\hspace{0.4cm} Por su parte las curvas obtenidas mediante las metodolog\'ias de Svensson sin optimizar y optmizada, son muy parecidas, s\'olo existe una peque\~na diferencia en el corto plazo. Sus alturas en comparaci\'on con la curva de los splines son m\'as elevadas, lo cual ocasiona que sus precios sean relativamente m\'as bajos.

\newpage

\begin{figure}[h]
  \scalebox{0.65}{\includegraphics{images/Comparativo_vebono.png}}
\caption{Curva VEBONO spline vs Svensson.}
\label{curva_spline_comp_veb}
\end{figure}


\newpage

\section{Conclusiones y recomendaciones.}

\hspace{0.4cm}La curva de rendimiento es una herramienta muy importante al momento de obtener informaci\'on acerca de la tasa de inter\'es o rendimiento a una fecha determinada ya que dicha curva relaciona la maduraci\'on o fecha de vencimiento con el rendimiento. Una de las aplicaciones de esta curva, es que a partir de ella es posible obtener con facilidad el precio de un determinado instrumento, bono o t\'itulo, lo cual es de suma utilidad al momento de querer realizar alguna operaci\'on con el mismo, ya sea compra o venta del instrumento, ya que se tendr\'ia de antemano un precio referencial a partir del cual se puede tomar una decisi\'on. En otras palabras, la curva de rendimiento es una herramienta muy \'util al momento de tomar decisiones al realizar alguna inversi\'on.



\hspace{0.4cm} Para determinar dicha curva, varias metodolog\'ias han sido desarrolladas. Existen dos grandes enfoques que permiten su c\'alculo, el primer enfoque se basa en el uso de las metodolog\'ias param\'etricas de estimaci\'on, las cuales se caracterizan por estar atadas a ciertos par\'ametros, su uso es muy frecuete. Entre ellas, principalmente destacan la metodolog\'ia de Nelson y Siegel introducida en 1987 $[2]$, la metodolog\'ia de Svensson desarrollada en 1994 $[3]$, entre otras.


\hspace{0.4cm}Por otra parte, el segundo enfoque se centra en el uso de las metodolog\'ias no param\'etricas, las cuales se caracterizan por su flexibilidad ya que ellas no se encuentran atadas a ning\'un par\'ametro espec\'ifico sino que trabajan directamente con los datos suministrados. Entre ellas, destacan la metodolog\'ia de redes neuronales $[9]$, splines de polinomios $[6]$, splines c\'ubicos suavizados $[7]$, entre otras. En el presenta trabajo se aplica la \'ultima metodolog\'ia mencionada.


\hspace{0.4cm} La principal raz\'on de aplicar la metodolog\'ia de los splines c\'ubicos suavizados, fu\'e el balance que se obtiene como la misma, ya que ella presenta un equilibrio entre el ajuste a los datos y la suavidad de la curva resultante. Lo cual, es \'util cuando la data presenta mucho ruido ya que esta metodolog\'ia no interpola los valores ingresados sino que ajusta una curva suave que presenta el menor error de ajuste posible. En el presente trabajo se emple\'o la data de los Tif y Vebonos, instrumentos de la deuda p\'ublica nacional venezolana para el a\~no 2016 y 2017. Para ambos intrumentos se encontraron una cantidad aceptable de operaciones a partir de las cuales se calcu\'o el rendimiento y as\'i a partir de dichos valores calcular la curva de rendimientos. Una vez obtenida la curva, se procede a estimar los precios de los instrumentos involucrados.

\hspace{0.4cm} Al realizar la comparaciones de los precios te\'oricos obtenidos mediante las metodolog\'ias de splines c\'ubicos de suavizado, Svensson no optimizado y Svensson optimizado, se puede afirmar que los precios te\'oricos que m\'as se asemejan a los precios promedio de cada tipo de instrumento, son aquellos obtenidos por la metodolog\'ia de Svensson optimizado. La raz\'on de esto se debe al proceso de optimizaci\'on que se lleva a cabo en dicha metodolog\'ia en donde se busca minimizar esta brecha. 


\hspace{0.4cm}De entrada esta es la principal fortaleza de esta metodolog\'ia, pero esto ocasiona tambi\'en que la misma dependa en demas\'ia de los precios promedio, lo cual en ocasiones no es del todo recomendable, puesto que en caso de considerar un precio promedio at\'ipico para alg\'un instrumento, la metodolog\'ia buscar\'a que los precios te\'oricos se asemejen a dicho precio promedio. De esta manera, la principal fortaleza de esta metodolog\'ia se puede convertir en una de sus debilidades.

\hspace{0.4cm} Teniendo en cuenta esto, la metodolog\'ia de los splines c\'ubicos de suavizado tiene una ventaja ya que la misma no depende del comportamiento de los precios promedio, sino que la misma al ser una metodolog\'ia no param\'etrica se basa en trabajar directamente con los datos. Lo cual en este caso, es trabajar a partir del comportamiento de las operacione que se realicen con cada tipo de instrumento para un horizonte temporal dado.

\hspace{0.4cm} De esta manera, la metodolog\'ia de los splines c\'ubicos de suavizado se considera una alternativa viable al momento de calcular los precios te\'oricos de los instrumentos de la deuda p\'ublica nacional para un tiempo determinado, con el fin de realizar la valoraci\'on de alg\'un portafolio y de esta manera saber si existen ganancia o perdidas por la tenencia de alg\'un instrumento. Como problema a resolver en futuros trabajos, se propone el c\'alculo del par\'ametro de suavizamiento de manera autom\'atica. 






\backmatter

\begin{thebibliography}{99}

\bibitem{Mm} {\sc Maita B. Miriam A.}, {\it Estimaci\'on de una curva de rendimientos para los bonos de la deuda p\'ublica interna en Venezuela}, (trabajo de grado Maestr\'ia en Administraci\'on de Empresas menci\'on Finanzas).  Universidad Cat\'olica Andr\'es Bello, Caracas, Venezuela, 2010.

\bibitem{NS} {\sc Nelson, C. y Siegel, A.} {\it Parsimonius Modeling of Yield Curves}. Journal of
Business, $60$: $473$-$489$, 1987.

\bibitem{Sv} {\sc Svensson, L.} {\it Estimating and Interpreting Forward Interest Rates: Sweden 1992-1994}, NBER Working Papers, 4871. Estocolmo: National Bureau of Economic Research, 1994.

\bibitem{HT} Hunt, B. y Terry, C. {\it Zero-Coupon Yield Curve Estimation: A Principal Component-Polynomial Approach}, Technical report 81. Sydney: University of Technology Sydney - School of Finance and Economics, 1998.

\bibitem{NW} Nocedal, J. y Wright, S. {\it Numerical Optimization}. New York: Springer-Verlag, 1999.


\bibitem{FG} Fan, J. y Gijbels, I. {\it Local Polynomial Modelling and Its Applications}. New York: Chapman and Hall, 1996.

\bibitem{FG} B.W. Silverman. {\it Some Aspects of the Spline Smoothing Approach to Non-Parametric Regression Curve Fitting}. Journal of the Royal Statistical Society. Series B (Methodological), Vol. 47, No. 1, 1(1985), pp. 1-52, 1984.


\bibitem{FG} Friedman, J. H. {\it A Variable Span Smoother}, Technical report 5. Standford:
Standford University - Departament of Statistics, 1984.


\bibitem{FG} Sharda, R. {\it Neural networks for the MS/OR analyst: An application bibliography}. Interfaces, 24(2): 116-130, 1994.



\end{thebibliography}



\end{document}
