\chapter{Antecedentes y motivaci\'on.}

\section{Introducci\'on.}

\hspace*{0.4 cm} La curva de redimientos es una herramienta ampliamente utilizada por las autoridades de los bancos centrales en sus decisiones de pol\'itica monetaria, as\'i como tambi\'en por los agentes privados en la planificaci\'on de sus inversiones en instrumentos financieros [1]. La misma tiene una importancia capital
para el mundo acad\'emico y pr\'actico desde el punto de vista econ\'omico y financiero, al reflejar el precio intertemporal del dinero.


\hspace*{0.4 cm} La curva de rendimientos es una representaci\'on grafica que muestra la relaci\'on que existe entre los rendimientos de una clase particular de t\'itulos valores y el tiempo que falta para su vencimiento, lo cual es conocido como la estructura temporal de la tasa de inter\'es (ETTI) para instrumentos con riesgo similar pero con diferentes plazos de maduraci\'on. La ETTI es un indicador de la evoluci\'on futura de los tipos de inter\'es y de inflaci\'on, adem\'as, la mayor\'ia de los activos financieros se valoran mediante este indicador, por lo cual tambi\'en se considera b\'asico en el dise\~no de estrategias de gesti\'on de riesgos y en la toma de decisiones de inversi\'on y financiaci\'on (Fern\'andez J.L., Robles M.D., 2005, p. 243). Existen cuatro formas que puede adoptar una curva de rendimientos:

\begin{itemize}
  \item Curva ascendente: generalmente, la curva de rendimientos tiene esta forma, lo que indica que los inversionistas requieren mayores rendimientos para vencimientos de m\'as largo plazo, es decir, que los rendimientos var\'ian directamente con los plazos. 
  \item Curva descendente: indica que los rendimientos disminuyen a medida que aumentan los plazos.
  \item Curva horizontal: indica que independientemente del plazo de vencimiento, los rendimientos son los mismos; para per\'iodos muy largos, todas las curvas de rendimientos tienden a aplanarse.
  \item Curva creciente y decreciente: es el reflejo de una situaci\'on en la que los rendimientos de corto y largo plazo son los mismos y los rendimientos de mediano plazo son los que var\'ian.
\end{itemize}

\hspace*{0.4 cm} Es de esperar que una pendiente negativa de la curva de rendimientos o curva invertida (tasas de largo plazo menores a las de corto plazo) indique expectativas de una recesi\'on futura y, por lo tanto, menores tasas de inter\'es futuras; esto se puede explicar ya que los rendimientos esperados contienen informaci\'on sobre los planes de consumo de los agentes. 

\hspace*{0.4 cm} Entre las teor\'ias que explican la pendiente de la curva de rendimientos, se encuentran:

\begin{itemize}
  \item La teor\'ia de la preferencia por la liquidez: consiste en que los inversionistas prefieren manejar t\'i tulos a corto plazo, pues \'estos tienen una sensibilidad menor a los cambios en las tasas de inter\'es y ofrecen una mayor flexibilidad en las inversiones si se compara con los t\'itulos de largo plazo. Adem\'as, los prestatarios prefieren deuda a largo plazo, pues la de corto plazo los expone al riesgo de hacer una refinanciaci\'on de la deuda en condiciones adversas. Ambas situaciones, generan entonces, tasas de corto plazo relativamente bajas. En su conjunto, estos dos grupos de preferencias implican que en condiciones normales existe una Prima de Riesgo por Vencimiento (PRV) que aumenta en funci\'on de los a\~nos de vencimiento, haciendo que la curva de rendimientos posea una pendiente ascendente (Douglas, 1988 pp 367-370).
  \item La teor\'ia de la segmentaci\'on del mercado: considera el mercado de renta fija como una serie de distintos mercados, los inversionistas y los emisores est\'an restringidos por el sector espec\'ifico de maduraci\'on. De acuerdo con esta teor\'ia, la curva de rendimientos refleja una serie de condiciones de oferta y demanda que crean una secuencia de precios de equilibrio de mercado (tasas de inter\'es) de los fondos (Douglas, 1988, pp. 368-369).
  \item La teor\'ia del H\'abitat Preferido: plantea que los inversionistas intentar\'an liquidar sus inversiones en el menor plazo posible mientras que los prestamistas querr\'an tomar un plazo más largo; por lo tanto, dado que no se encuentran oferta y demanda de fondos para un mismo plazo, algunos inversionistas o prestatarios se ver\'an motivados a cambiar el plazo de la inversi\'on o el financiamiento pero, para lograrlo, deben ser compensados con un premio por el riesgo cuyo tamano reflejar\'a la extesi\'on de la aversi\'on al riesgo.
  \item La Hip\'otesis de las Expectativas (HE): plantea que las tasas de inter\'es de largo plazo deben reflejar por completo la informaci\'on revelada por las futuras tasas de inter\'es de corto plazo esperadas (Yung-Shi Liau, Jack J.W. Yang, 2009, p.180), o sea que los tipos de largo plazo no son m\'as que una suma ponderada de los tipos de corto plazo esperados (Fern\'andez J.L., Robles M.D., 2005, p. 244). As\'i, se puede afirmar entonces que la HE es una teor\'ia que plantea que las tasas de inter\'es exclusivamente representan las tasas previstas en el futuro.
\end{itemize}


\hspace*{0.4 cm} Uno de los principales objetivos que se persigue mediante el uso de esta herramienta, es el de estimar los precios de los t\'itulos de la deuda p\'ublica nacional que posee una instituci\'on financiera en su portafolio de inversiones en un per\'iodo determinado. 


\section{Deuda P\'ublica Nacional}

\hspace*{0.4 cm} Deuda es la obligaci\'on que un sujeto tiene de reintegrar, satisfacer o pagar, especialmente dinero. P\'ublico, por otra parte, es un adjetivo que refiere a aquello perteneciente a toda la sociedad o que es com\'un del pueblo.

\hspace*{0.4 cm} La noci\'on de deuda p\'ublica hace menci\'on al conjunto de deudas que mantiene el Estado frente a otro pa\'is o particulares. Se trata de un mecanismo para obtener recursos financieros a trav\'es de la emisi\'on de t\'itulos de valores.

\hspace*{0.4 cm} El Estado, por lo tanto, contrae deuda p\'ublica para solucionar problemas de liquidez (cuando el dinero en caja no resulta suficiente para afrontar los pagos inmediatos) o para financiar proyectos a medio o largo plazo.

\hspace*{0.4 cm} La deuda p\'ublica puede ser contra\'ida por la administraci\'on municipal, provincial o nacional. Al emitir t\'itulos de valores y colocarlos en los mercados nacionales o extranjeros, el Estado promete un futuro pago con intereses según los plazos estipulados por el bono.

\hspace*{0.4 cm} La emisi\'on de deuda p\'ublica, al igual que la creaci\'on de dinero y los impuestos, son medios que tiene el Estado para financiar sus actividades. La deuda p\'ublica, de todos modos, tambi\'en puede utilizarse como un instrumento de la pol\'itica econ\'omica, de acuerdo a la estrategia escogida por las autoridades.

\hspace*{0.4 cm} Tendr\'iamos que hablar, por un lado, de tres tipos diferentes de deuda p\'ublica, aunque es cierto que hay diversas clasificaciones. As\'i, aquellos son los siguientes:

\begin{itemize}
  \item A corto plazo. Dentro de esta categor\'ia se encuentran las Letras del Tesoro y se identifica por el hecho de que tiene un plazo de vencimiento que no supera el a\~no.
  \item A medio plazo. Los bonos del Estado son, por su parte, los m\'aximos exponentes de esta clase de deuda p\'ublica que se suele utilizar para hacer frente a lo que ser\'an los gastos ordinarios que tiene aquel.
  \item A largo plazo. Como su propio nombre indica, este tipo de deuda tiene una duraci\'on muy larga, que se fijar\'a convenientemente, y que puede incluso llegar a ser perpetua. En su caso, se recurre a aquel para hacer frente a lo que ser\'ian gastos extraordinarios o para situaciones especiales.
\end{itemize}


\hspace*{0.4 cm} Es posible clasificar la deuda p\'ublica de distintas maneras. La deuda p\'ublica real es aquella compuesta por los t\'itulos que pueden ser adquiridos por los particulares, los bancos privados y el sector exterior. La deuda p\'ublica ficticia, en cambio, es la emisi\'on destinada al Banco Central del pa\'is, que es un organismo de la misma administraci\'on p\'ublica.





\hspace*{0.4 cm} Para ello es importante conocer las caracter\'isticas de los t\'itulos \'o instrumentos que existen en el mercado venezolano, entre ellos tenemos los siguientes,

\vspace{0.4cm}

\begin{itemize}
  \item T\'itulos de inter\'es fijo (TIF): son instrumentos que se caracterizan por tener una renta fija, y se emiten en moneda nacional.
  \item T\'itulos de inter\'es variable (VEBONO): se caracterizan por poseer una renta variable, e igual que los TIF se emiten en Bol\'ivares.
  \item Bonos PDVSA : son instrumentos emitidos en d\'olares.
\end{itemize}

\vspace{0.5cm}

\noindent cabe destacar que en el presente trabajo s\'olo  se considerar\'an aquellos instrumentos emitidos en Bol\'ivares.

\vspace{0.5cm}

\hspace*{0.4 cm}Asociado a cada t\'itulo hay un monto de dinero que se invierte, el cual es entregado al ente emisor por el t\'itulo en s\'i, existe tambi\'en una fecha de emisi\'on, y una fecha de vencimiento, d\'ia en el cual el ente emisor retorna el monto invertido inicialmente. Es importante destacar que estos t\'itulos pagan un inter\'es al portador cada tres meses, y esta representa la ganancia que tiene el inversionista.

\vspace{0.5cm}

\hspace*{0.4 cm} Con el fin de determinar en que t\'itulo es mejor invertir, es necesario conocer el rendimiento al vencimiento que posee dicho t\'itulo, este valor nos da un idea de cu\'al ser\'a el retorno que recibir\'a el portador del t\'itulo por la tenencia o compra del mismo. Para calcular este valor es necesario conocer la fecha de vencimiento del t\'itulo as\'i como su valor nominal y su precio.


 \hspace*{0.4 cm} A partir de la siguiente f\'ormula podemos hallar dicho valor,

\vspace{0.5cm}

\begin{center}

$\displaystyle{P_{t,m} = \sum_{i=1}^{m-1}{\frac{c}{(1+R_{t,i})^i} + \frac{1+c}{(1+R_{t,m})^m}} }$

\end{center}

\vspace{0.5cm}

\noindent donde $P_{t,m}$ representa el precio del t\'itulo al tiempo t, y que tiene un vencimiento de m a\~nos, c representa el cup\'on asociado al t\'itulo, el \'indice $i = 1,...,m$ representa cuantos cupones todav\'ia le quedan al t\'itulo por pagar antes de su vencimiento. Por su parte $R_{t,m}$ representa el rendimiento al vencimiento del t\'itulo en el tiempo t y que tiene un vencimiento de m a\~nos.

\vspace{0.5cm}

\hspace*{0.4 cm}A partir de la f\'ormula anterior podemos afirmar que para calcular el rendimiento al vencimiento de un t\'itulo es necesario conocer su precio es una fecha espec\'ifica, pero esto no siempre es posible, esto se debe a las caracter\'isticas del mercado venezolano ya que son pocos los t\'itulos que cotizan y por ende no se conoce su precio. Dicho precio puede conocerse a diario mediante la informaci\'on suministrada en la pesta\~na ``0-22" del documento ``resumersec" del Banco Central de Venezuela, este ente publica cada d\'ia las operaciones realizadas con estos instrumentos, en este documento se puede obtener el precio de aquellos t\'itulos que coticen ese d\'ia, el problema radica en conocer los precios de aquellos t\'itulos que no est\'en en dicho documento.



\vspace{0.5cm}

\hspace*{0.4 cm} La curva de rendimientos presenta emp\'iricamente una serie de dificultades, debido a que se construye a trav\'es de una serie de precios (tasas) de instrumentos financieros discontinuos en el tiempo que, por lo general, est\'an lejos de ser una curva suave. Para su estimaci\'on existen diversas metodolog\'ias, las param\'etricas y las no param\'etricas. Las metodolog\'ias param\'etricas se basan en modelos asociados a una familia funcional que obedece al comportamiento de alguna distribuci\'on de probabilidad, sobre la cual suponemos que las caracter\'isticas de la poblaci\'on de inter\'es pueden ser descritas. Es as\'i como, los modelos dise\~nados en este contexto, basados en regresi\'on, buscan describir el comportamiento de una variable de inter\'es con otras llamadas ex\'ogenas, a trav\'es de funciones de v\'inculo lineales o no lineales.


\section{Metodolog\'ias Param\'etricas}

\hspace*{0.4 cm} Estad\'isticamente, un modelo param\'etrico es una familia funcional que
obedece al comportamiento de alguna distribuci\'on de probabilidad,sobre la cual suponemos que las caracter\'isticas de la poblaci\'on de inter\'es
pueden ser descritas. Es as\'i como, los modelos dise\~nados en este contexto,
basados en regresi\'on, buscan describir el comportamiento de una
variable de inter\'es con otras llamadas ex\'ogenas, a trav\'es de funciones de
v\'inculo lineales o no lineales.

La curva de Nelson-Siegel

\hspace*{0.4 cm} Nelson y Siegel (1987) introducen un modelo param\'etrico para el ajuste
de los rendimientos hasta la madurez de los bonos del tesoro de Estados
Unidos que se caracteriza por ser parsimonioso y flexible en modelar
cualquier forma t\'ipica asociada con las curvas de rendimientos. La estructura
param\'etrica asociada a este modelo permite analizar el comportamiento
a corto y a largo plazo de los rendimientos y ajustar -sin
esfuerzos adicionales-, curvas mon\'otonas, unimodales o del tipo S.


\hspace*{0.4 cm} Una clase de funciones que genera f\'acilmente las formas usuales de las
curvas de rendimientos es la asociada con la soluci\'on de ecuaciones en
diferencia. La teor\'ia de expectativas sobre la estructura de las tasas de
inter\'es promueve la investigaci\'on en este sentido, dado que si las tasas
spot son producidas por medio de una ecuaci\'on diferencial, entonces las
tasas forward -siendo pron\'osticos-, ser\'an la soluci\'on de las ecuaciones
diferenciales. La expresi\'on param\'etrica propuesta por Nelson y Siegel
(1987) que describe las tasas forward es exhibida a continuaci\'on:


\begin{center}
$\displaystyle{f(m) = \beta_{0} + \beta_{1} e^{\frac{-m}{\tau}} +\beta_{2} \left(\frac{-m}{\tau}\right)e^{\frac{-m}{\tau}}}$
\end{center}

\noindent donde $m$ denota la madurez del activo y $\beta_{0}$, $\beta_{1}$, $\beta_{2}$ y $\tau$ los par\'ametros a ser
estimados. Puesto que las tasas spot pueden ser obtenidas a trav�s de tasas
forward por medio de la expresi\'on:

\begin{center}
$\displaystyle{s(m) = \int_{0}^{m}f(x)dx}$
\end{center}

\noindent la ecuaci\'on que determina las tasas spot $s(m)$ de activos con madurez m es dada por:

\begin{center}
$\displaystyle{s(m) = \beta_{0}+ \beta_{1}\frac{\left(1-e^\frac{-m}{\tau}\right)}{m/\tau} + \beta_{2} \left(\frac{\left(1-e^\frac{-m}{\tau}\right)}{m/\tau} -  e^\frac{-m}{\tau}\right)}$
\end{center}


\noindent cuya ecuaci\'on es lineal si conocemos $\tau$ .

\hspace*{0.4 cm} El valor l\'imite del rendimiento es $\beta_{0}$ cuando el plazo al vencimiento m es grande, mientras que, cuando el plazo al vencimiento m es peque\~no el
rendimiento en el l\'imite es $\beta_{0}+\beta_{1}$. Igualmente, los coeficientes del
modelo de tasas forward pueden ser interpretados como medidas de
fortaleza al corto, mediano y largo plazo. La contribuci\'on al largo plazo
es determinada por $\beta_{0}$, $\beta_{1}$ lo hace al corto plazo ponderado por la
funci\'on mon\'otona creciente (decreciente) $e^{\frac{-m}{\tau}}$ cuando $\beta_{1}$ es negativo
(positivo) y $\beta_{2}$ lo hace al mediano plazo ponderado por la funci\'on
mon\'otona creciente (decreciente) $(\frac{-m}{\tau}) e^{\frac{-m}{\tau}}$ cuando $\beta_{2}$ es negativo
(positivo). Una de las principales utilidades de la curva ha sido para
prop\'ositos de control de la pol\'itica monetaria.

\hspace*{0.4 cm} Consecuentemente, $s(m)$ ser\'a la ecuaci\'on utilizada para captar la relaci\'on
subyacente entre los rendimientos y los plazos al vencimiento o madurez,
sin recurrir a modelos m\'as complejos que involucren un mayor n\'umero
de par\'ametros. Adicionalmente, dado que la curva de Nelson-Siegel
proporciona tasas spot compuestas continuas, estas deben transformarse
en cantidades discretas, a trav\'es de la funci\'on de descuento.


\begin{center}
$\displaystyle{s_{d}(m) = e^{\frac{s(m)}{100}} - 1}$
\end{center}

La curva de Svensson

\hspace*{0.4 cm} En la curva de Nelson-Siegel se destaca que cada coeficiente del modelo
contribuye en el comportamiento de las tasas forward en el corto,
mediano y largo plazo; no obstante, Svensson (1994) propone una nueva
versi\'on de la curva de Nelson-Siegel donde un cuarto t\'ermino es incluido
para producir un efecto adicional y semejante al proporcionado por ??2:
$\beta_{3}(\frac{m}{\tau_{2}})e^{\frac{-m}{\tau_{2}}}$.

\hspace*{0.4 cm} En este caso, la funci\'on para describir la din\'amica de las tasas forward es,

\begin{center}
$\displaystyle{f(m) = \beta_{0} + \beta_{1} e^{\frac{-m}{\tau_{1}}} +\beta_{2} \left(\frac{-m}{\tau_{1}}\right)e^{\frac{-m}{\tau_{1}}} + \beta_{3}\left(\frac{-m}{\tau_{2}}\right)e^{\frac{-m}{\tau_{2}}} }$
\end{center}

\hspace*{0.4 cm} La curva spot de Svensson puede ser derivada a partir de la curva
forward en forma semejante a la descrita para el modelo de Nelson-
Siegel, obteniendo la siguiente expresi\'on:


\begin{center}
$\displaystyle{s(m) = \beta_{0}+ \beta_{1}\frac{\left(1-e^\frac{-m}{\tau_{1}}\right)}{m/\tau_{1}} + \beta_{2} \left(\frac{\left(1-e^\frac{-m}{\tau_{1}}\right)}{m/\tau_{1}} -  e^\frac{-m}{\tau_{1}}\right) + \beta_{3} \left(\frac{\left(1-e^\frac{-m}{\tau_{2}}\right)}{m/\tau_{2}} -  e^\frac{-m}{\tau_{2}}\right)}$
\end{center}

\hspace*{0.4 cm} La funci\'on de descuento tiene que ser utilizada con el fin de obtener las
tasas estimadas para cada d\'ia de negociaci\'on o trading. Svensson (1994)
propone estimar los par\'ametros de la curva cero cup\'on (curva spot),
minimizando una medida de ajuste tal como la suma de cuadrados del
error sobre los precios spot; sin embargo, enfatiza en que los precios
pueden llegar a ser mal ajustados para los activos de madurez corta. En
lugar de llevar el an\'alisis por este camino, propone estimar los
rendimientos fundamentado, principalmente, en que las decisiones de la
pol�tica econ\'omica se basan en el comportamiento de las tasas y que
obteniendo las tasas a trav\'es de la curva, los precios pueden ser
calculados una vez la funci\'on de descuento es evaluada. De esta manera,
los par\'ametros son escogidos minimizando la suma de cuadrados de la
diferencia entre los rendimientos observados y estimados por la curva.

\hspace*{0.4 cm} La estimaci\'on es realizada por medio de m\'axima verosimilitud, m\'inimos
cuadrados no lineales o el m\'etodo de momentos generalizados. En
muchos casos, como afirma Svensson (1994), el modelo de Nelson-
Siegel proporciona ajustes satisfactorios, aunque en algunos casos
cuando la estructura de las tasas de inter\'es es m�s compleja, el ajuste del
modelo de Nelson-Siegel es poco satisfactorio y el modelo de Svensson
logra desempe\~narse mejor.


Polinomios de componentes principales

\hspace*{0.4 cm} Hunt y Terry (1998) propone un ajuste de la curva de rendimientos
utilizando polinomios. Si frecuentemente la curva es especificada como,

\begin{center}
$\displaystyle{y(\tau) = \beta_{0} + \beta_{1}\tau +\beta_{2}\tau^2 +\beta_{3}\tau^3}$
\end{center}


\hspace*{0.4 cm} La cual puede captar todas la formas que puede asumir la curva, su
principal problema recae en el ajuste para aquellas tasas con per\'iodos de
vencimiento bastante largos. Aunque los autores conocen sobre las
propiedades de parsimonia y de ajuste asociados con la curva de Nelson-
Siegel, critican los problemas que acarrea la estimaci\'on de sus
par\'ametros, proponiendo el ajuste de la curva de polinomios, bajo
algunas modificaciones.

\hspace*{0.4 cm} Una transformaci\'on sobre el t\'ermino de plazos ($\tau$) que remueve la
inestabilidad asociada con las tasas a largo plazo del polinomio (5) es
sugerida. El modelo recomendado, siguiendo la notaci\'on de Hunt y
Terry (1998) es:

\begin{center}
$\displaystyle{y(\tau) = \beta_{0} + \sum_{i=1}^{p} \beta_{i} \frac{1}{(1+\tau)^i}} $
\end{center} 

\noindent donde

\begin{center}
$\displaystyle{y(0) = \sum_{i=0}^{p}\beta_{i} \hspace*{0.2 cm} y \hspace*{0.2 cm} y(\infty) = \beta_{0}   }$
\end{center} 


\hspace*{0.4 cm}Investigaciones relacionadas con curvas de rendimientos, han llegado a
la conclusi\'on que modelos con tres o cuatro par\'ametros son suficientes
para obtener un buen ajuste de los datos (Hunt 1995). Por tal motivo,
Hunt y Terry (1998) proponen restringir p a tres o cuatro. Aunque este
n\'umero de par�metros no necesariamente determina si realmente la
bondad de ajuste pueda llegar a ser satisfactoria, los autores proponen
utilizar componentes principales sobre los primeros p t\'erminos
polinomiales $1/(1 + \tau)$, con el fin de seleccionar $k<p$ variables, a ser
incluidas en la ecuaci\'on (6). Utilizar las componentes principales
proporcionar\'a un menor error de ajuste en comparaci\'on con (5),
debido a su capacidad para captar variabilidad. Una descripci\'on
detallada respecto al c\'alculo de las componentes principales en el
esquema polinomial es dada por Hunt y Terry (1998).


Polinomios trigonom\'etricos

\hspace*{0.4 cm} Las funciones trigonom\'etricas pueden ser utilizadas para capturar de
forma satisfactoria las distintas configuraciones que pueden asumir las
curvas de rendimientos. En este caso, el modelo puede ser descrito como
$y(\tau) = \beta_{0} + \beta_{1}cos(\gamma_{1}\tau) + \beta_{2}sen(\gamma_{2}\tau)$; donde ?? representa la duraci\'on o la
madurez del papel, en tanto que $\beta_{0}$, $\beta_{1}$, $\beta_{2}$, $\gamma_{1}$ y $\gamma_{2}$ son los par\'ametros
objeto de inter\'es. Cualquier metodolog\'ia de optimizaci\'on no lineal puede
ser utilizada para estimar los par\'ametros del modelo (Nocedal y Wright
1999). Aunque podr\'ia asumirse un par\'ametro de fase en el modelo, este
no es considerado por motivos de parsimonia.

\section{Metodolog\'ias no Param\'etricas}

\hspace*{0.4 cm} La regresi\'on no param\'etrica se ha convertido en los \'ultimos a\~nos en un
\'area de excesivo estudio, debido a sus ventajas relativas respecto a los
modelos de regresi\'on basado en funciones. Entre las caracter\'isticas m\'as
importantes de estos modelos tenemos, la flexibilidad en los supuestos y
el ajuste dirigido espec\'ificamente a trav\'es de los datos.


\hspace*{0.4 cm} Dentro de un marco estad\'istico supondremos que tenemos un conjunto
de n observaciones $(x_{i}, y_{i})$, $i= 1, 2,., n$, independientes, donde se intenta
establecer las relaciones existentes entre una respuesta y un conjunto de
variables explicativas de forma semejante a los modelos de regresi\'on
cl\'asica.


\hspace*{0.4 cm} El modelo que relaciona este conjunto de variables es dado por:

\begin{center}
$\displaystyle{y_{i} = m(x_{i}) + \epsilon_{i}}$
\end{center} 



\noindent donde la funci\'on $m(.)$ no espec\'ifica una relaci\'on param\'etrica, sino
permitir que los datos determinen la relaci\'on funcional apropiada. Bajo
estas condiciones la idea es que la media m(.) sea suave, suavidad que
puede controlarse acotando la segunda derivada, $|m''(x)|???M$, para todo
x y M una constante.

Regresi\'on Kernel

\hspace*{0.4 cm} El m\'etodo m\'as simple de suavizamiento es el suavizador Kernel. Un
punto x se fija en el soporte de la funci\'on $m(.)$ y una ventana de
suavizamiento es definida alrededor de x. Frecuentemente, la ventana de
suavizamiento es simplemente un intervalo de la forma $(x ??? h, x + h)$,
donde h es un par\'ametro conocido como bandwidth.

\hspace*{0.4 cm} La estimaci\'on Kernel es un promedio ponderado de las observaciones
dentro de la ventana de suavizamiento,

\begin{center}
$\displaystyle{\hat{m}(x) = \frac{\sum_{i=1}^{n} K(\frac{x_{i}-x}{h}) y_{i}}{\sum_{i=1}^{n} K(\frac{x_{i}-x}{h})}}$
\end{center}

\noindent donde $K(.)$ es la funci\'on Kernel de ponderaci\'on. La funci\'on Kernel es escogida
de tal forma que las observaciones m\'as pr\'oximas a x reciben mayor peso. Una
funci\'on frecuentemente utilizada es la bicuadr\'atica:

$$ K(x) = \left\{ % para la llave grandota
        \begin{tabular}{cc}
        	$(1-x^2)^2$ & si $-1 \leq x  \leq 1$ \\
        	$0$ & si $x \ge 1, \hspace*{0.2 cm} x<-1$ \\
        \end{tabular}
\right. $$



\hspace*{0.4 cm} Sin embargo, otro tipo de funciones de peso son utilizadas, tal como la
gaussiana, $K(x) = (2 \sqrt{\pi})^{-1} e^{\frac{-x^2}{2}}$ y la familia beta sim�trica $K(x) = \frac{(1-x^2)_{+}^{\gamma}}{Beta(0.5,\gamma+1)}, \hspace*{0.2 cm}\gamma = 0,1,...$















\hspace*{0.4 cm} Por su parte, las metodolog\'ias no param\'etricas se han convertido en los \'ultimos a\~nos en un
\'area de gran estudio debido a sus ventajas relativas respecto a los modelos de regresi\'on basado en funciones. Entre las caracter\'isticas m\'as importantes de estos modelos tenemos, la flexibilidad en los supuestos y el ajuste dirigido espec\'ificamente a trav\'es de los datos. Entre estas metodolog\'ias se destacan la de regresi\'on Kernel, polinomios locales, splines de polinomios (Fan y Gibels, $1996$ [6]), splines c\'ubicos suavizados (B.W. Silverman, $1985$ [7]), super suavizador de Friedmann (Friedmann, $1984$ [8]) y redes neuronales artificiales (Sharda, $1994$ [9]).

\vspace{0.5cm}

\hspace*{0.4 cm} En el siguiente trabajo se propone el uso de la metodolog\'ia de splines c\'ubicos suavizados, la cual posee la ventaja de contar un factor de penalizaci\'on el cu\'al es muy \'util al momento de tener un balance entre la suavidad de la curva y su bondad de ajuste. A grandes rasgos estas metodolog\'ias se basan en estimar la curva de rendimientos de dichos t\'itulos, curva que relaciona el rendimiento al vencimiento con la maduraci\'on o fecha de vencimiento, con el fin de estimar los precios de los t\'itulos a un d\'ia en espec\'ifico. De esta manera a partir de una determinada fecha es posible mediante estas metodolog\'ias estimar el rendimiento al vencimiento de un t\'itulo y por ende saber su precio estimado.

\newpage

\section{Objetivos.}

\subsection{Objetivos  generales del trabajo.}

\begin{itemize}
  \item Estimar la curva de rendimientos mediante el uso de los Splines C\'ubicos Suavizados.
\end{itemize}

\subsection{Objetivos espec\'ificos del trabajo.}

\begin{itemize}
  \item Generar un hist\'orico para los t\'itulos de tasa de inter\'es fija (TIF), pertenecientes a la deuda p\'ublica nacional (DPN).
  \item Estimar la curva de rendimientos para los TIF.
  \item Generar un hist\'orico para los t\'itulos de tasa de inter\'es variable (VEBONO), pertenecientes a la deuda p\'ublica nacional (DPN).
  \item Estimar la curva de rendimientos para los VEBONO.
  \item Estimar los precios de los TIF pertenecientes a un portafolio en un momento espec\'ifico.
  \item Estimar los precios de los VEBONO pertenecientes a un portafolio en un momento espec\'ifico.

\end{itemize}



