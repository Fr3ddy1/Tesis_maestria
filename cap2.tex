\chapter{Marco te\'orico.}

\section{Teor\'ia de interpolaci\'on.}

\hspace{0.4cm} En el subcampo matem\'atico del an\'alisis num\'erico, se denomina interpolaci\'on a la obtenci\'on de nuevos puntos partiendo del conocimiento de un conjunto discreto de puntos.


\hspace{0.4cm} En ciertos casos el usuario conoce el valor de una funci\'on $f(x)$ en una serie de puntos $x_{1}, x_{2},..., x_{N}$, pero no se conoce una 
expresi\'on anal\'itica de $f(x)$ que permita calcular el valor de la funci\'on para un punto arbitrario. Un ejemplo claro son las mediciones de laboratorio, donde se mide cada minuto un valor, pero se requiere el valor en otro punto que no ha sido medido. Otro ejemplo son mediciones de temperatura en la superficie
de la Tierra, que se realizan en equipos o estaciones meteorol\'ogicas y se necesita calcular la temperatura en un punto cercano, pero distinto al punto de medida.

\hspace{0.4cm} La idea de la interpolaci\'on es poder estimar $f(x)$ para un valor de x arbitrario, a partir de la construcci\'on de una curva o superficie que une los puntos donde se han realizado las mediciones y cuyo valor si se conoce. Se asume que el punto arbitrario x se encuentra dentro de los l\'imites de los puntos de medici\'on, en caso contrario se llamar\'ia extrapolaci\'on. 

\hspace{0.4cm} Un proceso de interpolaci\'on se realiza en dos etapas:

\begin{itemize}
  \item Hacer un ajuste de los datos disponibles con una funci\'on interpolante.
  \item Evaluar la funci\'on interpolante en el punto de inter\'es x.
\end{itemize}


\hspace{0.4cm} Este proceso en dos etapas no es necesariamente el m\'as 
eficiente. La mayor\'ia de algoritmos comienzan con un punto cercano $f(x_{i})$, y poco a poco van aplicando correcciones m\'as pequenas a medida que la 
informaci\'on de valores $f(xi)$ m\'as distantes son incorporadas. El procedimiento toma aproximadamente $O(N^{2})$ operaciones. Si la funci\'on tiene un comportamiento suave, la \'ultima correci\'on ser\'a la m\'as pequena y puede ser utilizada para estimar un l\'imite a rango de error.


\hspace{0.4cm} Dentro de las intepolaciones m\'as usadas podemos destacar,

\begin{itemize}
  \item Interpolaci\'on lineal
  \item Interpolaci\'on de Hermite
  \item Interpolaci\'on polin\'omica 
  \item Interpolaci\'on de Splines 

\end{itemize}

Interpolaci\'on lineal

\hspace{0.4cm} La interpolaci\'on lineal es un procedimiento muy utilizado para estimar los valores que toma una funci\'on en un intervalo del cual conocemos sus valores en los extremos $(x_{1}, f(x_{1}))$ y $(x_{2},f(x_{2}))$. Para estimar este valor utilizamos la aproximaci\'on a la funci\'on f(x) por medio de una recta $r(x)$ (de ah\'i el nombre de interpolaci\'on lineal, ya que tambi\'en existe la interpolaci\'on cuadr\'atica). 


\hspace{0.4cm} La expresi\'on de la interpolaci\'on lineal se obtiene del polinomio interpolador de Newton de grado uno. Aunque s\'olo existe un \'unico polinomio que interpola una serie de puntos, existen diferentes formas de calcularlo. Este m\'etodo es \'util para situaciones que requieran un n\'umero bajo de puntos para interpolar, ya que a medida que crece el n\'umero de puntos, tambi\'en lo hace el grado del polinomio.

\hspace{0.4cm} Existen ciertas ventajas en el uso de este polinomio respecto al polinomio interpolador de Lagrange. Por ejemplo, si fuese necesario a\~nadir alg\'un nuevo punto o nodo a la funci\'on, tan s\'olo habr\'ia que calcular este \'ultimo punto, dada la relaci\'on de recurrencia existente y demostrada anteriormente.

\hspace{0.4cm}El primer paso para hallar la f\'ormula de la interpolaci\'on es definir la pendiente de orden n de manera recursiva, as\'i,

\begin{itemize}
  \item $f_{0}(x_{i})$ es el t\'ermino i-\'esimo de la secuencia. 
  \item $\displaystyle{f_{1}(x_{0},x_{1}) = \frac{f_{0}(x_{1})-f_{0}(x_{0})}{x_{1}-x_{0}}}$
  \item $\displaystyle{f_{2}(x_{0},x_{1},x_{2}) = \frac{f_{1}(x_{1},x_{2})-f_{1}(x_{0},x_{1})}{x_{2}-x_{0}}}$
\end{itemize}

\hspace{0.4cm} En general,

\begin{center}
$\displaystyle{f_{i}(x_{0},x_{1},...,x_{i-1},x_{i}) = \frac{f_{i-1}(x_{1},...,x_{i-1},x_{i})-f_{i-1}(x_{0},x_{1},...,x_{i-1} )}{x_{i}-x_{0}}}$
\end{center}


\noindent donde  $\displaystyle{x_{i}-x_{j}}$ representa la distancia entre dos elementos.

\hspace{0.4cm} Puede apreciarse c\'omo en la definici\'on general se usa la pendiente del paso anterior, $f_{i-1}(x_{1},...,x_{i-1},x_{i})$, a la cual se le resta la pendiente previa de mismo orden, es decir, el sub\'indice de los t\'erminos se decrementa en 1, como si se desplazara, para obtener $f_{i-1}(x_{0},x_{1},...,x_{i-1})$.

\hspace{0.4cm} N\'otese tambi\'en que aunque el t\'ermino inicial siempre es $x_{0}$ este puede ser en realidad cualquier otro, por ejemplo, se puede definir $f_{1}(x_{i-1},x_{i})$ de manera an\'aloga al caso mostrado arriba.

\hspace{0.4cm} Una vez conocemos la pendiente, ya es posible definir el polinomio de grado n de manera tambi\'en recursiva, as\'i,

\begin{itemize}
  \item $p_{0}(x) = f_{0} (x_{0}) =x_{0}$.  Se define as\'i ya que este valor es el \'unico que se ajusta a la secuencia original para el primer t\'ermino. 
  \item $\displaystyle{p_{1}(x) = p_{0}(x) +  f_{1} (x_{0},x_{1}) (x-x_{0})}$.
  \item $\displaystyle{p_{2}(x) = p_{1}(x) +  f_{2} (x_{0},x_{1},x_{2}) (x-x_{0})(x-x_{1})}$.
\end{itemize}

\hspace{0.4cm} En general,

\begin{center}
$\displaystyle{p_{i}(x) = p_{i-1}(x) +  f_{i} (x_{0},x_{1},...,x_{i-1},x_{i}) \prod_{j=0}^{i-1}(x-x_{j})}$
\end{center}


Interpolaci\'on de Lagrange

\hspace{0.4cm} Empezamos con un conjunto de $n+1$ puntos en el plano (que tengan diferentes coordenadas x), $(x_{0}, y_{0}), (x_{1}, y_{1}), (x_{2}, y_{2}),...,(x_{n}, y_{n})$. As\'i, queremos encontrar una funci\'on polin\'omica que pase por esos $n+1$ puntos y que tengan el menor grado posible. Un polinomio que pase por varios puntos determinados se llama un polinomio de interpolaci\'on.

\hspace{0.4cm} Una posible soluci\'on viene dada por el  polinomio de interpolaci\'on de Lagrange. Lagrange public\'o su f\'ormula en 1795 pero ya hab\'ia sido publicada en 1779 por Waring y redescubierta por Euler en 1783.

\hspace{0.4cm} Dado un conjunto de $k + 1$ puntos $(x_{0}.x_{1}),...,(x_{k}.x_{k+1})$, donde todos los $x_{j}$ se asumen distintos, el polinomio interpolador en la forma de Lagrange es la combinaci\'on lineal,

\begin{center}
$\displaystyle{L(x) = \sum_{j=0}^{k} y_{j} l_{j}(x)}$
\end{center}


\noindent donde $ l_{j}(x)$ son las bases polin\'omicas de Lagrange ,

\begin{center}
$\displaystyle{l_{j}(x) = \prod_{i=0,i \neq j}^{k} \frac{x-x_{i}}{x_{j}-x_{i}}  }$
\end{center}

\hspace{0.4cm} La funci\'on que estamos buscando es una funci\'on polin\'omica $L(x)$ de grado $k$ con el problema de interpolaci\'on puede tener tan solo una soluci\'on, pues la diferencia entre dos tales soluciones, ser\'ia otro polinomio de grado $k$ a lo sumo, con $k+1$ ceros. Por lo tanto, $L(x)$ es el \'unico polinomio interpolador.


\hspace{0.4cm} Si se aumenta el n\'umero de puntos a interpolar (o nodos) con la intenci\'on de mejorar la aproximaci\'on a una funci\'on, tambi\'en lo hace el grado del polinomio interpolador as\'i obtenido, por norma general. De este modo, aumenta la dificultad en el c\'alculo, haci\'endolo poco operativo manualmente a partir del grado 4, dado que no existen m\'etodos directos de resoluci\'on de ecuaciones de grado 4, salvo que se puedan tratar como ecuaciones bicuadradas, situaci\'on extremadamente rara.

\hspace{0.4cm} La tecnolog\'ia actual permite manejar polinomios de grados superiores sin grandes problemas, a costa de un elevado consumo de tiempo de computaci\'on. Pero, a medida que crece el grado, mayores son las oscilaciones entre puntos consecutivos o nodos. Se podr\'ia decir que a partir del grado 6 las oscilaciones son tal que el m\'etodo deja de ser v\'alido, aunque no para todos los casos.

\hspace{0.4cm} Sin embargo, pocos estudios requieren la interpolaci\'on de tan s\'olo 6 puntos. Se suelen contar por decenas e incluso centenas. En estos casos, el grado de este polimonio ser\'ia tan alto que resultar\'ia inoperable. Por lo tanto, en estos casos, se recurre a otra t\'ecnica de interpolaci\'on, como por ejemplo a la Interpolaci\'on polin\'omica de Hermite o a los splines c\'ubicos. Otra gran desventaja, respecto a otros m\'etodos de interpolaci\'on, es la necesidad de recalcular todo el polinomio si se var\'ia el n\'umero de nodos.

\hspace{0.4cm} Aunque el polinomio interpolador de Lagrange se emplea mayormente para interpolar funciones e implementar esto f\'acilmente en una computadora, tambi\'en tiene otras aplicaciones en el campo del \'algebra exacta, lo que ha hecho m\'as c\'elebre a este polinomio, por ejemplo en el campo de los proyectores ortogonales.


Interpolaci\'on de Hermite

\hspace{0.4cm} El objetivo de esta interpolaci\'on Hermite es minimizar el error producido en la interpolaci\'on de Lagrange de la funci\'on $f(x)$ sobre el intervalo $[a, b]$ sin aumentar el grado del polinomio interpolador.


\hspace{0.4cm} Dados un entero no negativo N y  N + 1 puntos $(x_{0},..., x_{N})$ de la recta distintos dos a dos y los valores $f^{(j)} (x_{i})$, donde $0<i<N$ y $0<j<k_{i-1}$ de una funci\'on f y de sus derivadas, encontrar un polinomio de grado $m = (k_{0} +k_{1} +...+k_{n-1},)$ tal que,

\begin{center}
$\displaystyle{P^{(j)} (x_{i}) = f^{(j)} (x_{i}), para \hspace{0.2cm} 0<i<N \hspace{0.2cm}y \hspace{0.2cm} 0<j<k_{i-1}}$
\end{center}

\hspace{0.4cm} El problema de interpolaci\'on de Hermite tiene soluci\'on \'unica, que se llama polinomio interpolador de Hermite. En lugar de interpolar sobre un soporte de puntos (de Tchebycheff) donde en general se desconoce el valor de la funci\'on, de hace de otra manera, imponiendo unas condiciones al polinomio,

\begin{itemize}
  \item  Igualar el valor de la funci\'on en en los puntos del soporte, $P (x_{i}) = f^{(j)}$
  \item Igualar el valor de algunas derivadas de la funci\'on tambi\'en en los puntos del soporte, $P^{(j)} (x_{i}) = f^{(j)} (x_{i})$
\end{itemize}

\hspace{0.4cm} Por lo que podemos dejar el polinomio de Hermite de grado (n-1) expresado de la siguiente manera,

\begin{center}
$\displaystyle{L(x) = \sum_{j=0}^{k} y_{j} l_{j} (x) }$
\end{center}

\noindent donde $l_{j}(x)$ son los polinomios de la base de Lagrange.


Interpolaci\'on Polin\'omica

\hspace{0.4cm}Asumamos que se tiene una tabla con n puntos, $(x_{1},y_{1})...,(x_{n},y_{n})$, donde los valores $x_{i}$, para $i=1,...,n$ est\'an ordenados de forma creciente y todos ellos son distintos. Supongamos que dichos puntos se representan en un plano cartesiano y se quiere determinar una curva suave que interpole dichos valores. As\'i, se desea determinar una curva que est\'e definida para todos los $x$ y tome los valores correspondientes de $y$, esto es, que interpole todos los datos de la tabla. Cabe destacar que los puntos considerados se les conoce como nodos.


\vspace{0.5cm}

\hspace{0.4cm} La primera idea natural es usar una funci\'on polinomial que represente esta curva, la cual se puede representar como sigue,

\vspace{0.5cm}
\begin{center}

$\displaystyle{P_{n} = \sum_{i=0}^{n-1} a_{i}x^{i}}$

\end{center}

\noindent tal curva se le conoce como funci\'on polinomial interpoladora de grado n. N\'otese que en cada nodo se satisface que $P_{n}(x_{k})=y_{k}$, donde $k=1,...,n$.

\vspace{0.5cm}

\hspace{0.4cm} As\'i, se tiene que si la tabla es representada mediante una funci\'on subyacente $f(x)$ tal que $f(x_{k})=y_{k}$, para todo k, entonces esta funci\'on puede ser aproximada mediante $P_{n}$, en los puntos intermedios.


\vspace{0.5cm}

\hspace{0.4cm} No ser\'ia descabellado pensar que a medida que los nodos se incrementan, la aproximaci\'on ser\'ia cada vez mejor, lamentablemente esto no siempre es cierto, debido a que en el caso de tener una data con mucho ruido la interpolaci\'on no tendr\'ia mucho sentido ya que la varianza de los valores interpolados ser\'ia muy grande. En este caso, los polinomios interpoladores resultantes ser\'ian una mala representaci\'on de la funci\'on subyacente.

\vspace{0.5cm}

\hspace{0.4cm} Para evitar este fen\'omeno, puede ser de utilidad relajar la condici\'on de que $f(x)$ deber\'ia ser una funci\'on que interpole todos los valores dados y en su lugar usar un trozo de un polinomio local de interpolaci\'on. La funci\'on mediante la cual se logra esta interpolaci\'on se le conoce como spline.


\section{Splines.}

\vspace{1 cm}

\hspace{0.4cm} Una funci\'on spline $S(x)$ es una funci\'on que consta de trozos de polinomios unidos por ciertas condiciones de suavizado. Un ejemplo simple, es una funci\'on poligonal (spline de primer grado), la cual se forma por polinomios lineales unidos, los cuales se definen entre cada par de nodos. Entre los nodos $x_{j}$ y $x_{j+1}$ se define un spline de primer grado como sigue, \\

\begin{center}

$\displaystyle{S(x) = a_{j}x + b_{j} = S_{j}(x)}$

\end{center}

\vspace{0.5cm}

\noindent este spline es lineal. Usualmente $S(x)$ se define como $S_{1}(x)$ para $x<x_{1}$ y como $S_{n-1}(x)$ para $x>x_{n}$, donde $x_{1}$ y $x_{n}$ son nodos frontera.

\vspace{0.5cm}

\hspace{0.4cm} Un spline de segundo grado es una uni\'on de polinomios cuadr\'aticos tal que $S(x)$ y su derivada $S^{(1)}(x)$ son continuas. Por su parte un spline c\'ubico, se representa mediante la uni\'on de polinomios c\'ubicos con primera y segunda derivada continuas. Este spline debido a su flexibilidad es el m\'as usado en las aplicaciones.

\vspace{0.5cm}

\hspace{0.4cm}Formalmente un spline c\'ubico con nodos $x_{1},...x_{n}$ se define a partir de un conjunto de polinomios de la forma,\\

\begin{center}

$\displaystyle{S_{j}(x) = a_{j} + b_{j}x +c_{j}x^2 +d_{j}x^3}$
\end{center}


\vspace{0.5cm}

\noindent con $x_{j}<x<x_{j+1}$, sujeto a las siguientes condiciones,\\


\begin{center}

$\displaystyle{a_{j-1} + b_{j-1}x_{j} +c_{j-1}x_{j}^2 +d_{j-1}x_{j}^3 = a_{j} + b_{j}x_{j} +c_{j}x_{j}^2 +d_{j}x_{j}^3}$\\
$\displaystyle{ b_{j-1} +2c_{j-1}x_{j} +3d_{j-1}x_{j}^2 = b_{j} +2c_{j}x_{j} +3d_{j}x_{j}^2}$\\
$\displaystyle{ 2c_{j-1} +6d_{j-1}x_{j} = 2c_{j} +6d_{j}x_{j}}$\\
$\displaystyle{ c_{0} = d_{0} = c_{n} =d_{n}}$

\end{center}


\hspace{0.4cm}As\'i para n nodos, existen $4(n-1)$ variables y $4(n-1)-2$ restricciones. Las mismas se deben a la necesidad de que el spline c\'ubico sea igual en los valores dados en cada nodo. Las primeras tres restricciones aseguran que la funci\'on resultante en su primera y segunda derivada sean continuas en los nodos. La restricci\'on final significa que el spline c\'ubico es lineal en el punto inicial y final de la muestra. Sin embargo, es importante resaltar que el spline c\'ubico tiene tercera derivada discontinua en los nodos.

\hspace{0.4cm}Debido a que hacen falta dos restricciones de borde, estas se deben a\~nadir. As\'i  $S^{(2)}(x_{1}) = S^{(2)}(x_{n}) = 0$ son las restricciones faltantes, estan hacen referencia a que el spline sea un spline c\'ubico natural. Como se mencion\'o al inicio si se considera una interpolaci\'on polinomial global de un conjunto de datos con mucho ruido pueden surgir aproximaciones no deseables e inestables. En constrate, un spline c\'ubico de interpolaci\'on encaja perfectamente con la suavidad de la funci\'on subyacente.

\vspace{0.5cm}

\hspace{0.4cm} Otra caracter\'istica de los splines es que con la adici\'on de un par\'ametro s\'olo se aumenta la dimensionalidad del espacio de par\'ametros en una unidad, ya que tres de los cuatro par\'ametros est\'an restringidos. De igual forma, al incrementar el n\'umero de nodos los splines toman formas funcionales m\'as flexibles, lo cual muestra la relaci\'on entre el grado aproximaci\'on que se logra con el spline y el n\'umero de nodos que lo definen.

\vspace{0.5cm}

\hspace{0.4cm} Mientras que las funciones spline son una herramienta interesante para interpolar funciones suaves, encontrarlas num\'ericamente no es tarea f\'acil. Una manera eficiente y muy estable para generar los splines necesarios para aproximar la funci\'on subyacente $f(x)$, es usando las bases de los B-splines c\'ubicos.

\vspace{0.5cm}

\hspace{0.4cm} Supongamos que tenemos un conjunto infinito de nodos $...<x_{-2}<x_{-1}<x_{0}<x_{1}<x_{2}<...$, entonces el j-\'esimo B-spline de grado cero es igual a $B^{0}_{j}(x)=1$, si $x_{j} \leq x \leq x_{j+1}$ y $B^{0}_{j}(x)=0$ en otro caso. Con la funci\'on $B^{0}_{j}(x)$ como punto de partida se puede generar B-splines de grados mayores mediante la siguiente f\'ormula recursiva,\\

\begin{center}

$\displaystyle{B^{k}_{j}(x) = \frac{(x-x_{j})B^{k-1}_{j}(x)}{x_{j+k}-x_{j}} + \frac{(x_{j+k+1}-x)B^{k-1}_{j+1}(x)}{x_{j+k+1}-x_{j+1}}}$
\end{center}

\vspace{0.5cm}

\noindent para $k\geq 1$. As\'i un B-spline de grado $k$ se define como,\\

\begin{center}

$\displaystyle{S^{k}(x) = \sum_{j=-\infty}^{\infty} \theta^{k}_{j} B^{k}_{j-k}(x)}$
\end{center}

\vspace{0.5cm}

\hspace{0.4cm} Una buena interrogante ser\'ia el como se determina los coeficientes $\theta^{k}_{j}$ en la expresi\'on anterior. Note que los B-splines de grado positivo no son ortogonales y por ende no poseen una expresi\'on simple para sus coeficientes.


Sin embargo, los c\'alculos empleados para los B-splines interpoladores de grado cero y uno, son bastante sencillos,\\

\begin{center}

$\displaystyle{S^{0}(x) = \sum_{j=-\infty}^{\infty} y_{j} B^{0}_{j}(x),\hspace{0.4cm} S^{1}(x) = \sum_{j=-\infty}^{\infty} y_{j} B^{1}_{j-1}(x) }$
\end{center}

\vspace{0.5cm}

\hspace{0.4cm}Para splines de grados m\'as elevados, algunas arbitrariedades surgen al momento de calcular estos coeficientes. Por lo tanto, debido a que en las aplicaciones estadisticas existe un mayor inter\'es por encontrar una aproximaci\'on que una interpolaci\'on, la t\'ecnica de minimos cuadros puede ser empleada para calcular estos valores.


\vspace{0.5cm}

\hspace{0.4cm} Ahora bien, supongamos que se tiene un conjunto de $m$ funciones diferenciables $f(x)$, con soporte en el intervalo $[a,b]$, las cuales satifacen las siguientes condiciones,

% begin{itemize}
%   \item $f(x_{i})=y_{i}$, para i=1...,n
%   \item La m-1 derivada $f^{(m-1)}(x)$, es continua en x.
% \end{itemize}

\begin{itemize}
  \item[(i)] $f(x_{i})=y_{i}$, para $i=1...,n$.
  \item[(ii)] La m-1 derivada $f^{(m-1)}(x)$, es continua en x.
\end{itemize}

\hspace{0.4cm} El problema es encontrar entre todas esas funciones, una funci\'on tal que tenga la m\'inima integral del cuadro de su segunda derivada, esto es, una funci\'on que tenga el valor m\'as peque\~no de $\int_{a}^{b} (f^{(m)}(x))^2 dx$. Dicha funci\'on ser\'a la elecci\'on m\'as \'optima al momento de hallar un balance entre suavidad y ajuste de los datos.

\vspace{0.5cm}

\hspace{0.4cm} Se puede desmostrar que la soluci\'on de este problema es \'unica y la funci\'on en cuesti\'on es un spline polinomial que cumple la condici\'on i), y adem\'as satisface que,

\begin{itemize}
  \item[(a)] f es un pilinomio de grado no mayor que $m-1$ cuando $x \in [a,x_{1}]$ y $x \in [x_{n},b]$ .
  \item[(b)] F es un polinomio de grado no mayor a $2m-1$ para puntos interiores, $x \in [x_{i},x_{i+1}]$ con i=1,...,n.
  \item[(c)] f(x) tiene $2m-2$ derivadas continuas en el eje real.
\end{itemize}

\hspace{0.4cm} En resumen, la funci\'on $f$ m\'inima es un spline el cual est\'a conformado por trozos de polinomios unidos en los nodos $x_{i}$, donde dicha funci\'on tiene $2m-2$ derivadas continuas. N\'otese que en muchas aplicaciones $m=2$ es un valor muy utilizado y cuya soluci\'on viene dada mediante el spline c\'ubico natural.


\section{Regresi\'on no param\'etrica mediante splines de suavizado.}

\hspace{0.4cm} Consideremos el siguiente modelo de regresi\'on homoced\'astico,\\

\begin{center}

$\displaystyle{Y_{i}=f(X_{i})+\epsilon_{i}}, \hspace{0.3cm} para \hspace{0.2cm} i=1,...,n$
\end{center}

\vspace{0.5cm}

\noindent donde los $\epsilon_{i}$ son errores de media cero independientes e id\'enticamente distribuidos.

\vspace{0.5cm}

\hspace{0.4cm} Uno de los posibles m\'etodos para emplear splines es aproximar la funci\'on de regresi\'on subyacente mediante las bases de splines, por ejemplo, la base de los B-splines c\'ubicos. As\'i, se escoge una secuencia fija de nodos $-\infty<t_{1}<t_{2}<...<t_{J}<\infty$, los cuales pueden diferir de los predictores. Luego, se calculan los elementos de la base c\'ubica de spline correspondiente.

\vspace{0.5cm}

\hspace{0.4cm}Es posible mostrar que s\'olo son necesarios $J+4$ elementos de esta base. Denotemos a estos elementos por $B_{j}(x)$, as\'i el spline polinomial lo podemos expresar como sigue,

\begin{center}

$\displaystyle{S(x)=\sum_{j=1}^{J+4} \theta_{j}B_{j}(x)}$
\end{center}

\vspace{0.5cm}

\hspace{0.4cm}Entonces los coeficientes $\theta_{j}$ pueden ser calculados al ser considerados como los par\'ametros que se obtienen al minimizar la suma de los errores al cuadrado,

 \begin{center}

$\displaystyle{\sum_{i=1}^{n} \left[ Y_{i} - \sum_{j=1}^{J+4} \theta_{j}B_{j}(X_{j})\right]^2}$
\end{center}

\vspace{0.5cm}

\hspace{0.4cm}Denotamos por $\hat{\theta_{j}}$ al estimador de m\'inimos cuadrados y definimos el estimador del spline polinomial como sigue,

 \begin{center}

$\displaystyle{ \hat{f}_{n}(x) = \sum_{j=1}^{J+4} \hat{\theta_{j}}B_{j}(x)}$
\end{center}

\vspace{0.5cm} Otro enfoque, se basa en la idea de encontrar un curva suave que minimize la suma penalizada de errores al cuadrado, es decir, que minimize la siguiente expresi\'on,

\begin{equation}\label{min}
  n^{-1}\sum_{j=1}^{n}(Y_{j}-f(X_{j}))^2+\mu \int_{a}^{b} [f^{(m)} (x)]^2 dx
\end{equation}

\vspace{0.5cm}


\noindent para alg\'un $\mu > 0$. As\'i como el enfoque de interpolaci\'on anterior, la soluci\'on de este problema de minimizaci\'on es un spline, el cual recibe el nombre de estimador de spline de suavizado.

\vspace{0.5cm}

\hspace{0.4cm} En particular, para el caso $m=2$ el minimizador de (\ref{min}), es un spline c\'ubico natural. Note que $\mu$ juega el papel de par\'ametro de suavizado, este t\'ermino se puede interpretar como una penalizaci\'on por rugosidad de la funci\'on. Curvas que cambian lenta o suavemente presentan un valor peque\~no de la integral, por ejemplo, en una funci\'on lineal la integral toma el valor de cero.

\vspace{0.5cm}

\hspace{0.4cm} De hecho, la primera suma en (\ref{min}) penaliza la falta de fidelidad de la aproximaci\'on de la data mediante el spline. El segundo t\'ermino es el responsable de la suavidad de la aproximaci\'on obtenida mediante el spline. Para ver esto consid\'erese los casos extremos, es decir, cuando $\mu =0$ y $\mu=\infty$. El primer caso conduce a una interpolaci\'on, esto es $\hat{f}(X_{i})=Y_{i}$ para $i=1,...,n$. El otro caso, conduce a una regresi\'on lineal pues $f^{(2)}(x)\equiv 0$.

\vspace{0.5cm}

\hspace{0.4cm} Por lo tanto $\mu$ es el par\'ametro de suavizado que controla la medida del estimador del spline polinomial, el cual puede variar desde el modelo m\'as complicado e inestable hasta el modelo m\'as simple. En otras palabras, la ecuaci\'on (\ref{min}) representa un balance entre la fidelidad o ajuste de los datos, representado mediante la suma de los residuos al cuadrado y la suavidad de la curva resultante, la cual se representa por la integral del cuadrado de la m-\'emisa derivada.


\section{Citas}

%Para citar un libro o un art\'{\i}culo se hace as\'{\i} : \cite{ADRS} \cite{Ar}
