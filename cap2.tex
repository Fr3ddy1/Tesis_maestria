\chapter{Teoremas, Lemas, Proposiciones ... }

\section{Algunos comandos para un texto matem\'atico}

En los textos matem\'aticos se encuentran las siguientes estructuras:

Teorema, Proposici\'on, Lema, Corolario, Definici\'on, Ejemplo, Ejercicio, Notaci\'on,
Observaci\'on, Resumen, Demostraci\'on, Bibliograf\'{\i}a.

Para presentarlas en espa\~nol se deben dar algunas instrucciones como comandos de encabezado
que comienzan con theoremstyle, newtheorem.

Cuando queramos dar una definici\'on y una proposici\'on en el cuerpo del documento
procedemos como se indica a continuaci\'on.

\begin{definition} \index{espacio!m\'etrico}
Un \textit{espacio m\'etrico} es un par $(X, d)$ donde $X$ es un conjunto y $d: X \times X
\rightarrow {\mathbb R}$ es una funci\'on tal que:
\begin{itemize}
\item[(i)]   $d(x, y) \geq 0$  para todo $x, y \in X$, \item[(ii)]  $d(x, y) = 0$ si y s\'olo
si $x = y$, \item[(iii)]  $d(x, y) = d(y, x)$  para todo $x, y \in X$, \item[(iv)]  $d(x, z)
\leq d(x, y) + d(y, z)$  para todo $x, y, z \in X$.
\end{itemize}
\end{definition}

\

La instrucci\'on index tiene que ver con la creaci\'on del \'{\i}ndice anal\'{\i}tico.

\

En la definici\'on escribimos textit con la intenci\'on de obtener en it\'alicas el nombre
del objeto que se define.

\

\begin{proposition} \label{norma-met}
Sea $V$ un espacio vectorial y sea $\parallel \: \parallel$ una norma en $V$. Entonces  la
funci\'on $d$ definida por $$d(x,y) = \parallel x - y \parallel$$ es una m\'etrica en $V$.
\end{proposition}

\

Ejemplo de referencias cruzadas:

Note que hemos colocado label{norma-met} despu\'es del inicio de la Proposici\'on.   Esto es
para etiquetar esta porposici\'on y poder hacer referencia a ella posteriormente. Si queremos
hacer referencia al resultado anterior, usamos hablamos de la Proposici\'on
(\ref{norma-met}).


\section{Manejo de referencias cruzadas.}

A veces se quiere hacer referencia a una f\'ormula. En ese caso se debe etiquetar la
f\'ormula:

\begin{align}
y = ax + b \label{rec}
\end{align}

La ecuaci\'on (\ref{rec}) es la ecuaci\'on de una recta de pendiente $a$.

Las rectas paralelas a los ejes se pueden expresar con
\begin{align}
y & = b \label{rec-hor}\\
x & = c \label{rec-ver}
\end{align}
La ecuaci\'on (\ref{rec-hor}) es la ecuaci\'on de una recta horizontal y la ecuaci\'on
(\ref{rec-ver}) es la ecuaci\'on de una recta vertical.
